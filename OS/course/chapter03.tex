\chapter{物理内存管理:连续内存分配}
\section{计算机体系结构和内存层次}
内存管理:
\begin{enumerate}
	\item 抽象
	\subitem 逻辑地址空间
	\item 保护
	\subitem 独立地址空间
	\item 共享
	\subitem 访问相同的内存
	\item 虚拟化
	\subitem 更大的地址空间
\end{enumerate}
操作系统的内存管理方式
\begin{enumerate}
	\item 操作系统采用的内存管理方式
	\subitem 重定位(relocation)
	\subitem 分段(segmentation)
	\subitem 分页(paging)
	\subitem 虚拟存储(virtual memory)
	\subsubitem 目前多数系统采用按需页式虚拟存储
	\item 实现高度依赖硬件
	\subitem 与计算机存储架构紧耦合
	\subitem MMU(内存管理单元):处理CPU存储访问请求的硬件
\end{enumerate}

\section{地址空间和地址生成}
\subsection{地址空间的定义}
\begin{enumerate}
	\item 物理地址空间——硬件支持的地址空间
	\subitem 起始地址0,直到$MAX_{sys}$
	\item 逻辑地址空间——在CPU运行的进程看到的地址
	\subitem 起始地址0,直到$MAX_{prog}$
\end{enumerate}
\subsection{逻辑地址的生成}
高级语言$\rightarrow$汇编语言$\rightarrow$进一步汇编$\rightarrow$
链接,加上函数库$\rightarrow$重定位,程序加载
\subsection{地址生成时机和限制}
\begin{enumerate}
	\item 编译时
	\begin{enumerate}
		\item 假设起始地址已知
		\item 如果起始地址改变,必须重新编译
	\end{enumerate}
	\item 加载时
	\begin{enumerate}
		\item 如编译时起始位置未知,编译器需生成可重定位的代码(relocatable code)
		\item 加载时,生成绝对地址
	\end{enumerate}
	\item 执行时
	\begin{enumerate}
		\item 执行时代码可移动
		\item 需地址转换(映射)硬件支持
	\end{enumerate}
\end{enumerate}
\subsection{地址生成过程}
\begin{enumerate}
	\item CPU
	\begin{enumerate}
		\item ALU:需要逻辑地址的内存内容
		\item MMU:进行逻辑地址和物理地址的转换
		\item CPU控制逻辑:给总线发送物理地址请求
	\end{enumerate}
	\item 内存
	\begin{enumerate}
		\item 发送物理地址的内容给CPU
		\item 或接受CPU数据到物理地址
	\end{enumerate}
	\item 操作系统
	\subitem 建立逻辑地址LA和物理地址PA的映射
\end{enumerate}
\subsection{地址检查}
进程P逻辑地址空间$\rightarrow$CPU$\rightarrow$逻辑地址,端长度寄存器,检查内存是否异常(检查偏移量)$\rightarrow$如果正常,交给段基址寄存器,生成物理地址
$\rightarrow$到内存中找出数据\par
操作系统影响:段长度寄存器、段基址寄存器,设置起始(base)和最大(limit)
逻辑地址空间

\section{连续内存分配}
\subsection{连续内存分配和内存碎片}
\begin{itemize}
	\item 计算机体系结构/内存层次
	\subitem 给进程分配一块不小于指定大小的连续的物理内存区域
	\item 内存碎片
	\subitem 空闲内存不能被使用
	\item 外部碎片
	\subitem 分配单元之间的未被利用内存
	\item 内部碎片
	\subitem 分配单元内部的未被使用内存
	\subitem 取决于分配单元大小是否要取整
\end{itemize}
\subsection{连续内存分配:动态分区分配}
\begin{itemize}
	\item 动态分区分配
	\begin{enumerate}
		\item 当程序被加载执行时,分配一个进程指定大小可变的分区(块、内存块)
		\item 分区的地址是连续的
	\end{enumerate}
	\item 操作系统需要维护的数据结构
	\begin{enumerate}
		\item 所有进程的已分配分区
		\item 空闲分区(Empty-blocks)
	\end{enumerate}
	\item \textbf{动态分区分配策略}
	\begin{enumerate}
		\item 最先匹配(First-fit)
		\item 最佳匹配(Best-fit)
		\item 最差匹配(Worst-fit)
	\end{enumerate}
\end{itemize}
\subsection{最先匹配(First Fit Allocation)策略}
思路:分配n个字节,使用第一个可用的空间比n大的空闲块\\
原理\&实现
\begin{enumerate}
	\item 空闲分区列表按地址顺序排序
	\item 分配过程时,搜索一个合适的分区
	\item 释放分区时,检查是否可与临近的空闲分区合并
\end{enumerate}
优点
\begin{enumerate}
	\item 简单
	\item 在高地址空间有大块的空闲分区
\end{enumerate}
缺点
\begin{enumerate}
	\item 外部碎片
	\item 分配大块时较慢
\end{enumerate}
\subsection{最佳匹配(Best Fit Allocation)策略}
思路:分配n字节分区时,查找并使用不小于n的最小空闲分区\\
原理\&实现
\begin{enumerate}
	\item 空闲分区列表按照大小排序
	\item 分配时,差找一个合适的分区
	\item 释放时,查找并合并临近的空闲分区(如果找到)
\end{enumerate}
优点
\begin{enumerate}
	\item 大部分分配的尺寸较小时,效果很好
	\subitem 可避免大的空闲分区被拆分
	\subitem 可减少外部碎片的大小
	\subitem 相对简单
\end{enumerate}
缺点
\begin{enumerate}
	\item 外部碎片
	\item 释放分区较慢
	\item 容易产生很多无用的小碎片
\end{enumerate}
\subsection{最差匹配(Worst Fit Allocation)策略}
思路:分配n字节,使用尺寸不小于n的最大空闲分区\\
原理\&实现
\begin{enumerate}
	\item 空闲分区列表按照大小排序
	\item 分配时,选最大的分区
	\item 释放时,查找并合并临近的空闲分区,进行可能的合并,并调整空闲分区列表顺序
\end{enumerate}
优点
\begin{enumerate}
	\item 中等大小的分配较多时,效果最好
	\item 避免出现太多的小碎片
\end{enumerate}
缺点
\begin{enumerate}
	\item 外部碎片
	\item 释放分区较慢
	\item 容易破坏大的空闲分区,因此后续难以分配大的分区
\end{enumerate}
\section{碎片整理}
\subsection{碎片整理:紧凑(compaction)}
碎片整理:通过调整进程占用的分区位置来减少或避免分区碎片\par
碎片紧凑:通过分配给进程的内存分区,以合并外部碎片。\par
碎片紧凑的条件:所有的应用程序可动态重定位\par
注:什么时候移动,开销
\subsection{碎片整理:分区对换(Swapping in/out)}
分区对换:通过抢占并回收等待状态进程的分区,以增大可用内存空间。\par
在早期内存紧张时,可实现多进程交替。

\section{伙伴系统}
\subsection{伙伴系统(Buddy System)}
\begin{itemize}
	\item 约定整个可分配的分区的大小为$2^U$
	\item 需要的分区大小为$2^{U-1}<s\le2^U$时,把整个块分配给该进程
	\subitem 如$s\le2^{i-1}$,将大小为$2^i$的空闲分区划分为两个大小为
	$2^{i-1}-1$的空闲分区
	\subitem 重复切块过程,直到$2^{i-1}<s\le2^i$,并把一个空闲分区分配给该进程
\end{itemize}
\subsection{伙伴系统的实现}
\subsubsection{数据结构}
\begin{itemize}
	\item 空闲块按大小和起始地址组织成二维数组
	\item 初始状态:只有一个大小为$2^U$的空闲块
\end{itemize}
\subsubsection{分配过程}
\begin{itemize}
	\item 由小到大在空闲块数组中找最小的可用空闲块
	\item 如果空闲块过大,对可用空闲块进行二等分,直到得到合适的可用空闲块
\end{itemize}
\subsubsection{释放过程}
\begin{itemize}
	\item 把释放的块放入空闲块数组
	\item 合并满足合并条件的空闲块
\end{itemize}
合并条件
\begin{itemize}
	\item 大小相同$2^i$
	\item 地址相邻
	\item 起始地址较小的块的起始地址必须是$2^{i+1}$的倍数
\end{itemize}
\subsection{ucore中的物理内存管理}
\begin{lstlisting}
struct pmm_manager {
	const char *name;
	void (*init)(void);
	void (*init_memmap)(struct Page *base, size_t n);
	struct Page *(*alloc_pages)(size_t order);
	void (*free_pages)(struct Page *base, size_t n);
	size_t (*nr_free_pages)(void);
	void (*check)(void);
}
\end{lstlisting}
\section{问题}
\begin{enumerate}
\item 在启动页机制的情况下,在CPU运行的用户进程访问的地址空间是(B)
\begin{enumerate}[A]
	\item 物理地址空间
	\item 逻辑地址空间
	\item 外设地址空间
	\item 都不是
\end{enumerate}
用户进程访问的内存地址是虚拟地址

\item 在使能分页机制的情况下,更合适的外碎片整理方法是(C)
\begin{enumerate}[A]
	\item 紧凑(compaction)
	\item 分区对换(Swapping in/out)
	\item 都不是
\end{enumerate}	
分页方式不会有外碎片

\item 操作系统中可采用的内存管理方式包括(ABCD)
\begin{enumerate}[A]
	\item 重定位(relocation)
	\item 分段(segmentation
	\item 分页(paging)
	\item 段页式(segmentation+paging)
	
\end{enumerate}	


\item 连续内存分配的算法中,会产生外碎片的是(ABC)
\begin{enumerate}[A]
	\item 最先匹配算法
	\item 最差匹配算法
	\item 最佳匹配算法
	\item 都不会
\end{enumerate}	


\item 描述伙伴系统(Buddy System)特征正确的是(ABC)
\begin{enumerate}[A]
	\item 多个小空闲空间可合并为大的空闲空间
	\item 会产生外碎片
	\item 会产生内碎片
	\item 都不对
\end{enumerate}	
\end{enumerate}