\chapter{物理内存管理:非连续内存分配}
\section{非连续内存分配的需求背景}
\subsection{非连续分配的设计目标}
连续分配的特点:
\begin{enumerate}
	\item 分配给程序的物理内存必须连续;
	\item 存在外碎片和内碎片;
	\item 内存分配的动态修改困难;
	\item 内存利用率较低。
\end{enumerate}
非连续分配的设计目标:提高内存利用效率和管理灵活性。
\begin{enumerate}
	\item 允许一个程序的使用非连续的物理地址空间;
	\item 允许共享代码与数据;
	\item 支持动态加载和动态链接。
\end{enumerate}
非连续分配需要解决的问题:
\begin{enumerate}
	\item 如何实现虚拟地址和物理地址的转换;
	\subitem 软件实现(灵活,开销大)
	\subitem 硬件实现(够用,开销小)
\end{enumerate}
非连续分配的硬件辅助机制
\begin{enumerate}
	\item 如何选择非连续分配中的内存分块大小
	\subitem 段式存储管理(segmentation)
	\subitem 页式存储管理(paging)
\end{enumerate}

\section{段式存储管理}
\subsection{段地址空间}
进程的短地址空间由多个端组成
\begin{itemize}
	\item 主代码段
	\item 子模块代码段
	\item 公用库代码段
	\item 堆栈段(stack)
	\item 堆数据(heap)
	\item 初始化代码段
	\item 符号表等
\end{itemize}
段式存储管理的目的:更细粒度和灵活的分离和共享。
\subsection{段访问机制}
\subsubsection*{段的概念}
\begin{itemize}
	\item 段:表示访问方式和存储数据等属性相同的一段地址空间。
	\item 对应一个连续的内存“块”
	\item 若干个段组成进程逻辑地址空间
\end{itemize}
\textbf{段访问}:逻辑地址由二元组(s, addr)表示
\begin{itemize}
	\item s —— 段号
	\item addr —— 段内偏移
\end{itemize}
\textbf{段访问的硬件实现}:\par 
CPU加载程序后,根据逻辑地址的段号,查找段表(OS设置)的段描述符,找出基址和长度。然后硬件的MMU根据长度和偏移地址判断是否越界(内存异常),然后
MMU使用段基址+偏移量即可。

\section{页式存储管理}
\subsection{页式存储管理的概念}
\noindent\textbf{页帧}(物理页面,Frame,Page Frame)
\begin{itemize}
	\item 把物理地址空间划分为大小相同的基本分配单位
	\item 2的n次方,如512,4096,8192
\end{itemize}
\textbf{页面}(页,逻辑页面,Page)
\begin{itemize}
	\item 把逻辑地址空间划分为大小相同的基本分配单位
	\item 帧和页的大小必须是相同的
\end{itemize}
页面到页帧
\begin{itemize}
	\item 逻辑地址到物理地址的转换
	\item 页表
	\item MMU/TLB
\end{itemize}
\subsection{帧(Frame)}
帧:物理内存被划分为大小相等的帧。\\
内存物理地址的表示:二元组(f, o)
\begin{itemize}
	\item f——帧号(F位,共有$2^F$个帧)
	\item o——帧内偏移 (S位,每帧有$2^S$字节)
\end{itemize}
\begin{center}
	$\text{物理地址}=f\times2^S+o$
\end{center}
基于页帧的物理地址计算实例:\par
16-bit的地址空间\par
9-bit(512 byte)大小的页帧\par
物理地址=(3, 6)\par 
F=7, S=9, f=3, o=6\par 
实际地址:
$2^9\times3+6=1536+6=1542$
\subsection{页(Page)}
进程逻辑地址空间被划分为大小相等的页
\begin{itemize}
	\item 页内偏移=帧内偏移
	\item 通常:页号大小$\neq$帧号大小
\end{itemize}
进程逻辑地址的表示:二元组(p, o)
\begin{itemize}
	\item p——页号(p位,$2^P$个页)
	\item o——页内偏移(S位,每页有$2^S$字节)
\end{itemize}
$\text{虚拟地址}=p\times2^S+o$
\subsection{页式存储的地址映射——页表}
\begin{itemize}
	\item 页到帧的映射
	\item 逻辑地址中的页号是连续的
	\item 物理地址中的帧号是不连续的
	\item \textbf{不是所有的页}都有对应的帧
\end{itemize}
页表保存了逻辑地址——物理地址之间的映射关系\\
CPU加载程序获取逻辑地址(p, o),根据逻辑地址的页号,在页表中查找相应的
帧号,然后根据帧号和帧内偏移相加得到实际地址。
\section{页表概述}
\subsection{页表结构}
\begin{itemize}
	\item 每个进程都有一个页表
	\subitem 每个页面对应一个页表项
	\subitem 随进程运行状态而动态变化
	\subitem 页表基址寄存器(PTBR:Page Table Base Register)
\end{itemize}
页表项的组成:
\begin{itemize}
	\item 帧号:f
	\item 页表项标志
	\subitem 存在位(resident bit)
	\subitem 修改位(dirty bit)
	\subitem 引用位(clock/reference bit)
\end{itemize}
页表地址转换实例:\\
假定:具有16位地址的计算机系统,物理内存大小为:32KB,每页大小:1024字节
\subsection{页式存储管理机制的性能问题}
\begin{itemize}
	\item 内存访问性能问题
	\begin{enumerate}
		\item 访问一个内存单元需要2次内存访问
		\item 第一次访问:获取页表项
		\item 第二次访问:访问数据
	\end{enumerate}
\item 页表大小问题
\begin{enumerate}
	\item 页表可能非常大
	\item 64位机器如果每页1024字节,那么一个页表的大小会使多少?
\end{enumerate}
\item 如何处理?
\begin{enumerate}
	\item 缓存Caching
	\item 间接Indirection访问
\end{enumerate}
\end{itemize}
\section{快表和多级页表}
\subsection{快表Translation Look-aside Buffer,TLB}
缓存近期访问的页表项
\begin{itemize}
	\item TLB使用关联存储(associative memory)实现,具备快速访问性能
	\item 如果TLB命中,物理页号可以很快被获取
	\item 如果TLB未命中,对应的表项被更新到TLB中
\end{itemize}
\subsection{多级页表}
通过间接引用将页号分成k级
\begin{itemize}
	\item 建立页表树
	\item 减少每级页表的长度
\end{itemize}

\section{反置页表}
\subsection{大地址空间问题}
对于大地址空间(64-bits)系统,多级页表变得繁琐
\begin{itemize}
	\item 比如:5级页表
	\item 逻辑(虚拟)地址空间增长速度快于物理地址空间
\end{itemize}
页寄存器和反置页表的思路
\begin{itemize}
	\item 不让页表与逻辑地址空间的大小相对应
	\item 让页表与物理地址空间的大小相对应
\end{itemize}
\subsection{页寄存器Page Registers}
每个帧与一个页寄存器关联,寄存器内容包括:
\begin{itemize}
	\item 使用位(Residence bit):此帧是否被进程占用
	\item 占用页号(Occupier):对应的页号p
	\item 保护位(Protection bits)
\end{itemize}
页寄存器示例:
\begin{itemize}
	\item 物理内存大小:$4096\times=4K\times4KB=16MB$
	\item 页面大小$4096 bytes = 4KB$
	\item 页帧数$4096=4K$
	\item 页寄存器使用的空间(假设每个页寄存器占8字节)
	\subitem 8$\times$4096=32 Kbytes
	\item 页寄存器带来的额外开销
	\subitem 32K/16M=0.2\%
	\item 虚拟内存的大小:任意
\end{itemize}
页寄存器优点:
\begin{itemize}
	\item 页表大小相对于物理内存而言很小
	\item 页表大小与逻辑地址空间大小无关
\end{itemize}
页寄存器缺点:
\begin{itemize}
	\item 页表信息对调后,需要依据帧号可找页号
	\item 在页寄存器中搜索逻辑地址中的页号
\end{itemize}
\subsection{页寄存器的地址转换}
\begin{enumerate}
	\item CPU生成的逻辑地址如何找对应的物理地址?
	\begin{itemize}
		\item 对逻辑地址进行Hash映射,以减少搜索范围
		\item 需要解决可能的冲突
	\end{itemize}
\item 用快表缓存页表项后的页寄存器搜索步骤:
\begin{itemize}
	\item 对逻辑地址进行Hash变换
	\item 在快表中查找对应页表项
	\item 有冲突时遍历冲突项链表
	\item 查找失败时,产生异常
\end{itemize}
\item 快表的限制
\begin{itemize}
	\item 快表的容量限制
	\item 快表的功耗限制(Strong ARM上快表功耗占27\%)
\end{itemize}
\end{enumerate}
\subsection{反置页表}
基于Hash映射值查找对应页表项中的帧号。
\begin{itemize}
	\item 进程标识与页号的Hash值可能冲突
	\item 页表项中保护位、修改位、访问位和存在位等标识
\end{itemize}
\subsection{反置页表的Hash冲突}
匹配下一项
\section{段页式存储管理}
\subsection{段页式存储管理的需求}
\begin{itemize}
	\item 段式存储在内存保护方面有优势,页式存储在内存利用和优化转移到后备存储方面有优势
\end{itemize}
\subsection{段页式存储管理实现}
在段式存储管理基础上,给每个段加一级页表。
\par 段页式存储管理中的内存共享
\begin{itemize}
	\item 通过指向\textbf{相同的页表基址},实现进程间的段共享
\end{itemize}

\section{问题}
\begin{enumerate}
\item 描述段管理机制正确的是(ABCD)
\begin{enumerate}[A]
	\item 段的大小可以不一致
	\item 段可以有重叠
	\item 段可以有特权级
	\item 段与段之间是可以不连续的
\end{enumerate}

\item 描述页管理机制正确的是(ABC)
\begin{enumerate}[A]
	\item 页表在内存中
	\item 页可以是只读的
	\item 页可以有特权级
	\item 都不对
\end{enumerate}

\item 页表项标志位包括(ABCD)
\begin{enumerate}[A]
	\item 存在位(resident bit)
	\item 修改位(dirty bit)
	\item 引用位(clock/reference bit)
	\item 只读位(read only OR read/write bit)
\end{enumerate}

\item 可有效应对大地址空间可采用的页表手段是(AB)
\begin{enumerate}[A]
	\item 多级页表
	\item 反置页表
	\item 页寄存器方案
	\item 单级页表
\end{enumerate}
页寄存器和反置页表很像,但它们的一个区别是进程ID在地址转换中的使用。没有进程ID(也就是说页寄存器方案)时,页表占用的空间仍然是与进程数相关的(也就是每个进程对应一组页寄存器?)。反置页表的大小只与物理内存大小,与并发进程数无关。\\
采用页寄存器的硬件开销会很大。所以现在的通用CPU(包括64位的CPU)没有采用这种方式,大部分还是多级页表。由于有TLB作为缓存,效率还不错。

\end{enumerate}