\chapter{虚拟存储概念}
\section{虚拟存储的需求背景}
\begin{enumerate}
	\item 增长迅速的存储需求:程序规模的增长速度远远大于存储器容量的增长
\end{enumerate}
存储层次结构
\begin{itemize}
	\item 理想中的存储器
	\subitem 容量更大、速度更快、价格更便宜的非易失性存储器
	\item 实际中的存储器:使用存储结构解决
\end{itemize}
\subsection{操作系统的存储抽象}
操作系统对存储的抽象:地址空间
\subsection{虚拟存储需求}
\begin{itemize}
	\item 计算机系统时常出现内存空间不够用
	\begin{enumerate}
		\item 覆盖overlay
		\subitem 应用程序\textbf{手动}把需要的指令和数据保存在内存中
		\item 交换swapping
		\subitem 操作系统\textbf{自动}把暂时不能执行的程序保存到外存中
		\item 虚拟存储
		\subitem 在有限容量的内存中,以页为单位\textbf{自动装入更多更大}的程序
	\end{enumerate}
\end{itemize}

\section{覆盖和交换}
\subsection{覆盖技术}
目标:在较小的可用内存中运行较大的程序
\par 方法:依据程序逻辑结构,将程序划分为若干\textbf{功能相对独立}的模块,将不会同时执行的模块\textbf{共享同一块内存区域}
\begin{itemize}
	\item 必要部分(常用功能)的代码和数据常驻内存
	\item 可选部分(不常用部分)放在其他程序模块中,只在需要用到时装入内存
	\item 不存在调用关系的模块可相互覆盖,共用同一块内存区域
\end{itemize}
覆盖技术的不足:
\begin{enumerate}
	\item 增加编程困难
	\begin{itemize}
		\item 需要程序员划分功能模块,并确定模块间的覆盖关系
		\item 增加了编程的复杂度
	\end{itemize}
	\item 增加执行时间
	\begin{itemize}
		\item 从外存装入覆盖模块
		\item 时间换空间
	\end{itemize}
\end{enumerate}
\subsection{交换技术}
目标:增加正在运行或需要运行的程序的内存。
\par 实现方法:
\begin{itemize}
	\item 可将暂时不能运行的程序放到外存
	\item 换入换出的基本单位
	\subitem 整个进程的地址空间
	\item 换出(swap out)
	\subitem 把一个进程的整个地址空间保存到外存
	\item 换入(swap in)
	\subitem 将外存中某进程的地址空间读入到内存
\end{itemize}
交换技术面临的问题
\begin{itemize}
	\item 交换时机:何时需要发生交换?
	\subitem 只当内存空间不够时或有不够的可能时换出
	\item 交换区大小
	\subitem 存放所有用户进程的所有内存映像的拷贝
	\item 程序换入时的重定位:换出后再换入时要放回原处吗?
	\subitem 采用动态地址映射的方法
\end{itemize}
\subsection{覆盖和交换的比较}
\begin{enumerate}
	\item 覆盖
	\begin{itemize}
		\item 只能在没有调用关系的模块间
		\item 程序员须给出模块间的逻辑覆盖结构
		\item 发生在运行程序的内部模块间
	\end{itemize}
	\item 交换
	\begin{itemize}
		\item 以进程为单位
		\item 不需要模块间的逻辑覆盖结构
		\item 发生在内存进程间
	\end{itemize}
\end{enumerate}

\section{局部性原理}
虚拟存储技术的目标
\begin{enumerate}
	\item 只把部分程序放到内存中,从而运行比物理内存大的程序。
	\subitem 由操作系统自动完成,无需程序员的干涉;
	\item 实现进程在内存与外存之间的交换,从而获得更多的空闲内存空间
	\subitem 在内存与外存之间只交换进程的部分内容
\end{enumerate}
\subsection{局部性原理(principle of locality)}
局部性原理:程序在执行过程中的一个较短时期,所执行的指令地址和指令的操作数地址,分别局限于一定区域。
\begin{enumerate}
	\item 时间局部性
	\subitem 一条指令的一次执行和下次执行,一个数据的一次访问和下次访问
	都集中在一个较短的时期内
	\item 空间局部性
	\subitem 当前指令和邻近的几条指令,当前访问的数据和邻近的几个数据都
	集中在一个较小的区域内
	\item 分支局部性
	\subitem 一条跳转指令的两次执行,很可能跳到相同的内存位置
	\item 局部性原理的意义
	\subitem 从理论上来说,虚拟存储技术是能够实现的,而且可取得满意的结果
\end{enumerate}
程序编写:二维数组的遍历顺序。

\section{虚拟存储概念}
虚拟存储思路:
\begin{itemize}
	\item 将不常用的部分内存块暂存到外存中
\end{itemize}
虚拟存储原理:
\begin{enumerate}
	\item 装载程序时
	\subitem 只将当前指令执行需要的部分页面或段装入内存
	\item 指令执行中需要的指令或数据不在内存(称为缺页或缺段)时
	\subitem 处理器通知操作系统将相应的页面或段调入内存
	\item 操作系统将内存中暂时不用的页面或段保存到外存中
\end{enumerate}
实现方式
\begin{itemize}
	\item 虚拟页式存储
	\item 虚拟段式存储
\end{itemize}
虚拟存储的基本特征:
\begin{enumerate}
	\item 不连续性
	\subitem 物理内存分配非连续
	\subitem 虚拟地址空间使用非连续
	\item 大用户空间
	\subitem 提供给用户的虚拟内存可大于实际的物理内存
	\item 部分交换
	\subitem 虚拟存储只对部分虚拟地址空间进行调入和调出
\end{enumerate}
\subsection{虚拟存储的支持技术}
\begin{enumerate}
	\item 硬件
	\subitem 页式或短时存储中的地址转换机制
	\item 操作系统
	\subitem 管理内存和外存间页面或段的换入和换出
\end{enumerate}

\section{虚拟页式存储}
\subsection{虚拟页式存储管理}
\begin{itemize}
	\item 在页式存储管理的基础上,增加请求调页和页面置换
	\item 思路
	\begin{enumerate}
		\item 当用户程序要装载到内存运行时,只装入部分页面,就启动
		程序执行
		\item 进程在运行中发现有需要的代码或数据不在内存时,则向系统发出缺页异常请求
		\item 操作系统在处理缺页异常时,将外存中相应的页面调入内存,使得进程能继续运行
	\end{enumerate}
\end{itemize}
虚拟页式存储在页表中加一个无效项,当查询无效时,交给操作系统处理,然后由
操作系统进行查找缺页。
\subsection{虚拟页式存储中的页表项结构}
页表结构:\\
逻辑页号+访问位+修改位+保护位+驻留位+物理页帧号
\begin{enumerate}
	\item 驻留位:表示该页是否在内存
	\subitem 1表示该页位于内存中,该页表项是有效的,可以使用
	\subitem 0表示该页当前在外存中,访问该页表项将导致缺页异常
	\item 修改位:表示在内存中的该页是否被修改
	\subitem 回收该物理页面时,据此判断是否要把它的内容写回外存
	\item 访问位:表示该页面是否被访问过(读或写)
	\subitem 用于页面置换算法
	\item 保护位:表示该页的允许访问方式
	\subitem 只读、可读写、可执行
\end{enumerate}

\section{缺页异常}
\subsection{缺页异常的处理流程}
\begin{enumerate}[A]
	\item load M,查找页表
	\item 页表驻留位为0,缺页异常
	\item 操作系统缺页异常服务处理例程查找在外存中的页面
	\item 查找到后,页面换入
	\item 页表项修改
	\item 重新执行导致异常的指令
\end{enumerate}
A.在内存中有空闲物理页,分配一物理页帧,转至E步\\
B.依据页面置换算法选择将被替换的物理页帧f,对应逻辑页q\\
C.如果q被修改过,则把它写回外存\\
D.修改q的页表项中驻留位置为0\\
E.将需要访问的页p装入到物理页面f\\
F.修改p的页表项驻留位为1,物理页帧号为f。\\
G.重新执行产生缺页的指令。
\subsection{虚拟页式存储中的外存管理}
在何处保存未被映射的页?
\begin{itemize}
	\item 应能方便地找到在外存中的页面内容
	\item 交换空间(磁盘或者文件)
	\subitem 采用特殊格式存储未被映射的页面
\end{itemize}
虚拟页式存储的外存选择
\begin{itemize}
	\item 代码段:可执行二进制文件(不放交换空间)
	\item 动态加载的共享库程序段:动态调用的库文件(不放交换空间)
	\item 其他段:交换空间
\end{itemize}
\subsection{虚拟页式存储管理的性能}
\begin{enumerate}
	\item 有效存储访问时间(effective memory access time EAT)
	\\ $EAT=\text{访存时间}\times(1-p)+\text{缺页异常处理时间}\times \text{缺页率}p$
\end{enumerate}
例子:访存时间10ns,磁盘访问时间5ms,缺页率p,页修改率q
\par $EAT=10\times(1-p)+5,000,000\times p\times (1+q)$