\chapter{页面置换算法}
\section{页面置换算法的概念}
\subsection{置换算法的功能和目标}
\noindent 功能:
\begin{itemize}
	\item 当出现缺页异常,需调入新页面而内存已满时,置换算法\textbf{选择被置换的物理页面}
\end{itemize}
设计目标:
\begin{itemize}
	\item 尽可能\textbf{减少页面的调入调出次数}
	\item 把未来不再访问或短期内不访问的页面调入
\end{itemize}
页面锁定(frame locking)
\begin{itemize}
	\item 描述必须常驻内存的逻辑页面
	\item 操作系统的关键部分
	\item 要求响应速度的代码和数据
	\item 页表中的锁定标志位(lock bit)
\end{itemize}
\subsection{置换算法的评价方法}
\begin{enumerate}
	\item 记录进程访问内存的页面轨迹;\\
	虚拟地址访问用(页号,位移)表示:(3,0),(1,9),(4,1),(2,1),(5,3),
	(2,0),(1,9),(2,4),(3,1),(4,8)\\
	页面轨迹:3,1,4,2,5,2,1,2,3,4$\rightarrow$c,a,d,b,e,b,a,b,c,d
	\item 评价方法
	\subitem 模拟页面置换行为,记录产生缺页的次数
	\subitem 更少的缺页,更好的性能
\end{enumerate}
\subsection{页面置换算法分类}
\begin{enumerate}
	\item 局部页面置换算法
	\subitem 置换页面的选择范围仅限于当前进程占用的物理页面内
	\subitem 最优算法、先进先出算法、最近最久未使用算法
	\subitem 时钟算法、最不常用算法
	\item 全局置换算法
	\subitem 置换页面的选择范围是所有可换出的物理页面
	\subitem 工作计算法、缺页率算法
\end{enumerate}

\section{最优算法、先进先出算法和最近最久未使用算法}
\subsection{最优页面置换算法(OPT,optimal)}
基本思路:置换在未来最长时间不访问的页面\par 
算法实现:
\begin{itemize}
	\item 缺页时,计算内存中每个逻辑页面的下一次访问时间
	\item 选择\textbf{未来最长时间不访问的页面}
\end{itemize}
算法特征:
\begin{itemize}
	\item 缺页最少,是理想情况
	\item 实际系统中\textbf{无法实现}
	\item 无法预知每个页面在下次访问前的等待时间
	\item 作为置换算法的性能评价依据
	\subitem 在模拟器上运行某个程序,并记录每一次的页面访问情况
	\subitem 第二遍运行时使用最优算法
\end{itemize}
最优页面置换算法示例
\subsection{先进先出算法(First-In First-Out,FIFO)}
思路:选择\textbf{在内存驻留时间最长}的页面进行置换。
\par 实现
\begin{itemize}
	\item 维护一个记录所有位于内存中的逻辑页面链表
	\item 链表元素\textbf{按驻留内存的时间排序},链首最长,链尾最短
	\item 出现缺页时,选择链首页面进行置换,新页面加到链尾
\end{itemize}
特征
\begin{itemize}
	\item 实现简单
	\item 性能较差,调出的页面可能是经常访问的
	\item 进程分配物理页面数增加时,缺页并不一定减少(Belady现象)
	\item 很少单独使用
\end{itemize}
\subsection{最近最久未使用算法(Least Recently Used,LRU)}
思路:
\begin{itemize}
	\item 选择\textbf{最长时间没有被引用的}页面进行置换
	\item 如某些页面长时间未被访问,则它们在将来还可能会长时间不会访问
\end{itemize}
实现:
\begin{itemize}
	\item 缺页时,计算内存中每个逻辑页面的\textbf{上一次}访问时间;
	\item 选择\textbf{上一次使用到当前时间最长的页面}
\end{itemize}
特征
\begin{itemize}
	\item 最优置换算法的一种近似
\end{itemize}
\subsection{LRU算法的可能实现方法}
\begin{enumerate}
	\item 页面链表
	\subitem 系统维护一个按最近一次访问时间排序的页面链表
	\subsubitem 链表首节点是最近刚刚使用过的页面
	\subsubitem 链表尾结点是最久未使用的页面
	\subitem 访问内存时,找到相应页面,并把它移到链表之首
	\subitem 缺页时,置换链表尾结点的页面
	\item 活动页面栈
	\subitem 访问页面时,将此页号压入栈顶,并栈内相同的页号抽出
	\subitem 缺页时,置换栈底的页面
\end{enumerate}
特征:实现开销大。
\subsection{问题}
\begin{enumerate}
\item 物理页帧数量为3,虚拟页访问序列为 0,1,2,0,1,3,0,3,1,0,3,请问采用最优置换算法的缺页次数为(D)
\begin{enumerate}[A]
	\item 1
	\item 2
	\item 3
	\item 4
\end{enumerate}

\item 物理页帧数量为3,虚拟页访问序列为 0,1,2,0,1,3,0,3,1,0,3,请问采用LRU置换算法的缺页次数为(D)
\begin{enumerate}[A]
	\item 1
	\item 2
	\item 3
	\item 4
\end{enumerate}

\item 物理页帧数量为3,虚拟页访问序列为 0,1,2,0,1,3,0,3,1,0,3,请问采用FIFO置换算法的缺页次数为(D)
\begin{enumerate}[A]
	\item 1
	\item 2
	\item 3
	\item 6
\end{enumerate}
\end{enumerate}

\section{时钟置换算法和最不常用算法}
\subsection{时钟置换算法(Clock)}
思路:仅对页面的访问情况进行大致统计,又称最近未用(Not Recently Used, NRU)算法。\\
数据结构:
\begin{enumerate}
	\item 在页表项中增加\textbf{访问位},描述页面在过去一段时间的内访问情况
	\item 各页面组织成\textbf{环形链表}
	\item \textbf{指针}指向最先调入的页面
\end{enumerate}
算法
\begin{enumerate}
	\item 访问页面时,在页表项记录页面访问情况
	\item 缺页时,从指针处开始顺序查找未被访问的页面进行置换
\end{enumerate}
特征:时钟算法是LRU和FIFo的折中。\\
时钟置换算法的实现:
\begin{itemize}
	\item 页面装入内存时,访问位初始化为0;
	\item 访问页面(读/写)时,访问位置1
	\item 缺页时,从指针当前位置顺序检查环形链表
	\subitem 访问位为0,则置换该页
	\subitem 访问位为1,则访问位置0,并指针移动到下一个页面,直到找到可置换的页面
\end{itemize}
\subsection{改进的Clock算法}
思路:减少修改页的缺页处理开销。\\
算法:
\begin{enumerate}
	\item 在页面中增加修改位,并在访问时进行相应修改
	\item 缺页时,修改页面标志位,以跳过有修改的页面
\end{enumerate}
\begin{table}[!h]
	\centering
	\begin{tabular}{|c|c|c|c|}
		\hline
		指针扫过前&&指针扫过后&\\
		\hline
		使用位&修改位&使用位&修改位\\
		\hline
		0&0&置换&置换\\
		\hline
		0&1&0&0\\
		\hline
		1&0&0&0\\
		\hline
		1&1&0&1\\
		\hline
	\end{tabular}
\end{table}
\subsection{最不常用算法(Least Frequently Used,LFU)}
思路:缺页时,置换访问次数最少的页面\\
实现
\begin{enumerate}
	\item 每个页面设置为一个访问计数
	\item 访问页面时,访问计数加1
	\item 缺页时,置换计数最小的页面
\end{enumerate}
特征
\begin{enumerate}
	\item 算法开销大
	\item 开始时频繁使用,但以后不使用的页面很难置换
	\subitem 解决方法:计数定期右移
\end{enumerate}
LRU和LFU的区别:
\begin{itemize}
	\item LRU关注\textbf{多久未访问},时间越短越好
	\item LFU关注\textbf{访问次数},次数越多越好
\end{itemize}
\subsection{问题}
\begin{enumerate}
\item 物理页帧数量为4,虚拟页访问序列为 0,3,2,0,1,3,4,3,1,0,3,2,1,3,4 ,请问采用CLOCK置换算法(用1个bit表示存在时间)的缺页次数为(B)
\begin{enumerate}
	\item 8
	\item 9
	\item 10
	\item 11
\end{enumerate}

\item 物理页帧数量为4,虚拟页访问序列为 0,3,2,0,1,3,4,3,1,0,3,2,1,3,4 ,请问采用CLOCK置换算法(用2个bit表示存在时间)的缺页次数为(C)
\begin{enumerate}
	\item 8
	\item 9
	\item 10
	\item 11
\end{enumerate}
实际为7次
\end{enumerate}

\section{BELADY现象和局部置换算法比较}
\subsection{Belady现象}
现象:采用FIFO等算法时,可能出现分配的物理页面增加,缺页反而升高的异常现象。
\par 原因:
\begin{itemize}
	\item FIFO算法的置换特征与进程访问内存的动态特征矛盾
	\item 被它置换出去的页面并不一定是进程近期不会访问的
\end{itemize}
\par 思考:哪些置换算法没有Belady现象?
\begin{itemize}
	\item FIFO算法有Belady现象
	\item LRU算法没有Belady现象
\end{itemize}
\subsection{LRU、FIFO和Clock的比较}
\begin{enumerate}
	\item LRU算法和FIFO本质上都是先进先出的思路
	\subitem LRU依据页面的最近访问时间排序
	\subitem LRU需要动态调整顺序
	\subitem FIFO依据页面进入内存的事件排序
	\subitem FIFO的页面进入时间使固定不变的
	\item LRU可退化成FIFO
	\subitem 如页面进入内存后没有被访问,最近访问时间与内存的时间相同
	\item LRU算法性能较好,但系统开销较大
	\item FIFO算法系统开销较小,会发生Belady现象
	\item Clock算法是它们的折中
	\subitem 页面访问时,不动态调整页面在链表中的顺序,仅做标记
	\subitem 缺页时,再把它移动到链表末尾
	\item 对于未访问的页面,Clock和LRU算法的表现一样好
	\item 对于被访问过的页面,Clock算法不能记录准确访问顺序,而LRU算法可以
\end{enumerate}
\subsection{问题}
\begin{enumerate}
	\item 虚拟页访问序列为 1,2,3,4,1,2,5,1,2,3,4,5,物理页帧数量为3和4,采用FIFO置换算法,请问是否会出现bealdy现象(A)
	\begin{enumerate}[A]
		\item 会
		\item 不会
	\end{enumerate}
\item 下面哪些页面淘汰算法不会产生Belady异常现象(AB)
\begin{enumerate}[A]
	\item 先进先出页面置换算法(FIFO) 
	\item 时钟页面置换算法(CLOCK) 
	\item 最佳页面置换算法(OPT) 
	\item 最近最少使用页面置换算法(LRU)
\end{enumerate}
\end{enumerate}

\section{工作集置换算法}
局部置换算法没有考虑进程访存差异:
\begin{itemize}
	\item FIFO页面置换算法:假设初始顺序a$\rightarrow$b$\rightarrow$c
\end{itemize}
\subsection{全局置换算法}
思路:全局置换算法为进程分配\textbf{可变数目}的物理页面
\par 全局置换算法要解决的问题:
\begin{enumerate}
	\item 进程在不同阶段的内存需求是变化的
	\item 分配给进程的内存也需要在不同阶段有所变化
	\item 全局置换算法需要确定分配给进程的物理页面数
\end{enumerate}
CPU利用率和并发进程数的关系:
\begin{itemize}
	\item CPU利用率与并发进程数存在相互促进和制约的关系
	\subitem 进程数少时,提高并发进程数,可提高CPU利用率
	\subitem 并发进程导致内存访问增加
	\subitem 并发进程的内存访问会降低访存的局部性特征
\end{itemize}
\subsection{工作集}
工作集:一个进程当前正在使用的逻辑页面集合,可表示为二元函数$W(t,\Delta)$
\begin{itemize}
	\item t是当前的执行时刻
	\item $\Delta$称为工作集窗口(working-set window),即一个定长的的页面访问时间窗口
	\item $W(t,\Delta)$是指在当前时刻t前的$\Delta$时间窗口中的所有访问页面所组成的集合
	\item $|W(t,\Delta)|$指工作集的大小,即页面数目
\end{itemize}
\subsubsection*{工作集的变化}
\begin{itemize}
	\item 进程开始执行后,随着访问新页面逐步建立稳定的工作集
	\item 当内存访问的局部性区域的位置\textbf{大致稳定}时,工作集大小也大致稳定
	\item 局部性区域的位置改变时,工作集\textbf{快速扩张和收缩}过渡到下一个稳定值
\end{itemize}
\subsection{常驻集}
常驻集:在当前时刻,进程实际驻留在内存中的页面集合。
\par 工作集与常驻集的关系:
\begin{itemize}
	\item 工作集是进程在运行过程中固有的性质
	\item 常驻集取决于系统分配给进程的物理页面数目和页面置换算法
\end{itemize}
\par 缺页率与常驻集的关系
\begin{itemize}
	\item 常驻集$\supseteq$工作集,缺页减少
	\item 工作集发生剧烈变动(过渡)时,缺页较少
	\item 进程常驻集大小达到一定数目后,缺页率也不会明显下降
\end{itemize}
\subsection{工作集置换算法}
工作集置换算法思路:换出不在工作集中的页面
\par 窗口大小$\tau$
\begin{itemize}
	\item 当前时刻前$\tau$个内存访问的页引用是工作集,$\tau$被称为窗口大小
\end{itemize}
实现方法
\begin{itemize}
	\item 访存链表:维护窗口内的访存页面链表
	\item 访存时,换出不在工作集的页面;更新链表
	\item 缺页时,换入页面;更新访存链接
\end{itemize}
\subsection{问题}
\begin{enumerate}
	\item 物理页帧数量为5,虚拟页访问序列为 4,3,0,2,2,3,1,2,4,2,4,0,3,请问采用工作集置换算法(工作集窗口T=4)的缺页次数为()
	\begin{enumerate}[A]
		\item 2
		\item 3
		\item 4
		\item 5
	\end{enumerate}
实际为8
\end{enumerate}


\section{缺页率置换算法}
\subsection{缺页率(page fault rate)}
缺页率:缺页平均时间间隔的倒数$=\dfrac{\text{缺页次数}}{\text{内存访问次数}}$\\
影响缺页率的因素
\begin{itemize}
	\item 页面置换算法
	\item 分配给进程的物理页面数目
	\item 页面大小
	\item 程序的编写方法
\end{itemize}

\subsection{缺页率置换算法(PFF,Page-Fault-Frequency)}
通过调节常驻集大小,使每个进程的缺页率保持在一个合理的范围内。
\begin{itemize}
	\item 若进程缺页率过高,则增加常驻集以分配更多的物理页面
	\item 若进程缺页率过低,则减少常驻集以减少它的物理页面
\end{itemize}
缺页率置换算法的实现
\begin{itemize}
	\item 访存时,设置引用位标志
	\item 缺页时,计算从上次缺页时间$t_{last}$到现在$t_{current}$的时间间隔
	\begin{itemize}
		\item 如果$t_{current}-t_{last}>T$,则置换所有在$[t_{last},t_{current}]$时间内没有被引用的页
		\item $t_{current}-t_{last}\le T$,则增加缺失页到常驻集中
	\end{itemize}
\end{itemize}
\subsection{问题}
\begin{enumerate}
	\item 物理页帧数量为5,虚拟页访问序列为 4,3,0,2,2,3,1,2,4,2,4,0,3,请问采用缺页率置换算法(窗口T=2)的缺页次数为()
	\begin{enumerate}[A]
		\item 2
		\item 3
		\item 4
		\item 5
	\end{enumerate}
\end{enumerate}
\section{抖动和负载控制}
\subsection{抖动问题thrashing}
抖动
\begin{itemize}
	\item 进程物理页面太少,不能包含工作集
	\item 造成大量缺页,频繁置换
	\item 进程运行速度变慢
\end{itemize}
产生抖动的原因:随着驻留内存的进程数目增加,分配给每个进程的物理页面数不断减小,缺页率不断上升\\
操作系统需在并发水平和缺页率之间达到一个平衡。
\begin{itemize}
	\item 选择一个适当的进程数目和进程需要的物理页面数
\end{itemize}
\subsection{负载控制}
\begin{itemize}
	\item 通过调节并发进程数(MPL)来进行系统负载控制
	\subitem $\sum WSi=$内存大小
	\subitem 平均缺页间隔时间(MTBF=缺页异常处理时间(PFST))
\end{itemize}
\begin{itemize}
	\item MPL-multiprogramming level
	\item MTBF-mean time between page faults
	\item PFST-page fault service time
\end{itemize}