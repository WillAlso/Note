\chapter{一个简单的Web服务器}
\section{HTTP}
超文本传输协议(HyperText Transfer Protocol)是基于“请求-响应”的协议。
\par HTTP使用可靠地TCP连接,默认使用TCP 80端口。
\par Web服务器不负责联系客户端或建立到一个客户端的回调连接,客户端或服务端可提前关闭连接。
\subsection{HTTP请求}
一个HTTP请求包含:
\begin{itemize}
	\item 请求方法——统一资源标识符(Uniform Resource Identifier,URI)——协议/版本
	\item 请求头
	\item 实体
\end{itemize}
\par HTTP1.1支持7种请求方法:GET、POST、HEAD、OPTIONS、PUT、DELETE和TRACE。
统一资源定位符(Uniform Resource Locator,URL)实际上是URI的一种类型。
\par 请求头包含了客户端环境、请求实体正文的相关信息。各个请求头之间使用回车/换行(Carriage Return/LineFeed,CRLF)间隔开。
\par 在请求头和请求实体正文之间有一个空行,该行只有CRLF符。
\subsection{HTTP响应}
HTTP响应三部分:
\begin{itemize}
	\item 协议——状态码——描述
	\item 响应头
	\item 响应式体段
\end{itemize}
\par 响应头和响应实体正文之间只包含了CRLF的一个空行分隔。
\section{Socket类}
套接字使应用程序可以从网络中读取数据,可以向网络中写入数据。
\par Java中的套接字:java.net.Socket,构造函数:
\par \qquad public Socket(java.lang.String host, int port)
\par host为远程主机名称或IP地址,参数port是远程应用程序端口号。两者通信:
\begin{lstlisting}
Socket socket = new Socket("127.0.0.1", 8080);
OutputStream os = socket.getOutputStream();
boolean autoflush = true;
PrintWriter out = new PrintWriter(socket.getOutputStream(), autoflush);
BufferedReader in = new BufferedReader(new InputStreamReader(socket.getInputStream()));

out.println("GET /index.jsp HTTP/1.1");
out.println("Host: localhost:8080");
out.println("Connection: Close");
out.println();

boolean loop =true;
StringBuffer sb = new StringBuffer(8096);
while (loop) {
	if (in.ready()) {
		int i = 0;
		while (i != -1) {
			i = in.read();
			sb.append((char) i);
		}
		loop = false;
	}
	Thread.currentThread().sleep(50);
}
System.out.println(sb.toString());
socket.close();
\end{lstlisting}
\subsection{ServerSocket类}
与Socket不同,当服务器套接字接收到连接请求后,会创建Socket实例来处理与客户端通信。
\par ServerSocket提供4个构造函数,典型如下:
\begin{lstlisting}
public ServerSocket(int port, int backlog, InetAddress bindingAddress)
\end{lstlisting}
Port是服务器端口,IP地址一般是127.0.0.1(绑定地址),baklog表示在服务器拒绝接受传入的请求之前,传入的连接请求的最大队列长度。
\par 这里IP必须是java.net.InetAddress的实例,可以使用静态方法获取:
\begin{lstlisting}
InetAddress.getByName("127.0.0.1")
\end{lstlisting}
当创建ServerSocket实例后,等待传入的连接请求,可以通过ServerSocket的accept方法完成,只有收到请求后才返回,返回值是一个Socket实例。
\section{应用程序}
发送指定目录静态资源,但不发送任何头信息,程序包含三个部分:
\begin{itemize}
	\item HttpServer
	\item Request
	\item Response
\end{itemize}
\subsection{HttpServer}

\subsection{Request}
\subsection{Response}