\chapter{一个简单的Web服务器}
\section{HTTP}
超文本传输协议(HyperText Transfer Protocol)是基于“请求-响应”的协议。
\par HTTP使用可靠地TCP连接,默认使用TCP 80端口。
\par Web服务器不负责联系客户端或建立到一个客户端的回调连接,客户端或服务端可提前关闭连接。
\subsection{HTTP请求}
一个HTTP请求包含:
\begin{itemize}
	\item 请求方法——统一资源标识符(Uniform Resource Identifier,URI)——协议/版本
	\item 请求头
	\item 实体
\end{itemize}
\par HTTP1.1支持7种请求方法:GET、POST、HEAD、OPTIONS、PUT、DELETE和TRACE。
统一资源定位符(Uniform Resource Locator,URL)实际上是URI的一种类型。
\par 请求头包含了客户端环境、请求实体正文的相关信息。各个请求头之间使用回车/换行(Carriage Return/LineFeed,CRLF)间隔开。
\par 在请求头和请求实体正文之间有一个空行,该行只有CRLF符。
\subsection{HTTP响应}
HTTP响应三部分:
\begin{itemize}
	\item 协议——状态码——描述
	\item 响应头
	\item 响应式体段
\end{itemize}
\par 响应头和响应实体正文之间只包含了CRLF的一个空行分隔。
\section{Socket类}
套接字使应用程序可以从网络中读取数据,可以向网络中写入数据。
\par Java中的套接字:java.net.Socket,构造函数:
\par \qquad public Socket(java.lang.String host, int port)
\par host为远程主机名称或IP地址,参数port是远程应用程序端口号。两者通信:
\begin{lstlisting}
Socket socket = new Socket("127.0.0.1", 8080);
OutputStream os = socket.getOutputStream();
boolean autoflush = true;
PrintWriter out = new PrintWriter(socket.getOutputStream(), autoflush);
BufferedReader in = new BufferedReader(new InputStreamReader(socket.getInputStream()));

out.println("GET /index.jsp HTTP/1.1");
out.println("Host: localhost:8080");
out.println("Connection: Close");
out.println();

boolean loop =true;
StringBuffer sb = new StringBuffer(8096);
while (loop) {
	if (in.ready()) {
		int i = 0;
		while (i != -1) {
			i = in.read();
			sb.append((char) i);
		}
		loop = false;
	}
	Thread.currentThread().sleep(50);
}
System.out.println(sb.toString());
socket.close();
\end{lstlisting}
\subsection{ServerSocket类}
与Socket不同,当服务器套接字接收到连接请求后,会创建Socket实例来处理与客户端通信。
\par ServerSocket提供4个构造函数,典型如下:
\begin{lstlisting}
public ServerSocket(int port, int backlog, InetAddress bindingAddress)
\end{lstlisting}
Port是服务器端口,IP地址一般是127.0.0.1(绑定地址),baklog表示在服务器拒绝接受传入的请求之前,传入的连接请求的最大队列长度。
\par 这里IP必须是java.net.InetAddress的实例,可以使用静态方法获取:
\begin{lstlisting}
InetAddress.getByName("127.0.0.1")
\end{lstlisting}
当创建ServerSocket实例后,等待传入的连接请求,可以通过ServerSocket的accept方法完成,只有收到请求后才返回,返回值是一个Socket实例。
\section{应用程序}
发送指定目录静态资源,但不发送任何头信息,程序包含三个部分:
\begin{itemize}
	\item HttpServer
	\item Request
	\item Response
\end{itemize}
\subsection{HttpServer}
\begin{lstlisting}
public class HttpServer {
	
	/** WEB_ROOT is the directory where our HTML and other files reside.
	*  For this package, WEB_ROOT is the "webroot" directory under the working
	*  directory.
	*  The working directory is the location in the file system
	*  from where the java command was invoked.
	*/
	public static final String WEB_ROOT =
	System.getProperty("user.dir") + File.separator  + "webroot";
	
	// shutdown command
	private static final String SHUTDOWN_COMMAND = "/SHUTDOWN";
	
	// the shutdown command received
	private boolean shutdown = false;
	
	public static void main(String[] args) {
		HttpServer server = new HttpServer();
		server.await();
	}
	
	public void await() {
		ServerSocket serverSocket = null;
		int port = 8080;
		try {
			serverSocket =  new ServerSocket(port, 1, InetAddress.getByName("127.0.0.1"));
		}
		catch (IOException e) {
			e.printStackTrace();
			System.exit(1);
		}
		
		// Loop waiting for a request
		while (!shutdown) {
			Socket socket = null;
			InputStream input = null;
			OutputStream output = null;
			try {
				socket = serverSocket.accept();
				//收到请求获取java.io.*的实例
				input = socket.getInputStream();
				output = socket.getOutputStream();
				
				// create Request object and parse
				Request request = new Request(input);
				request.parse();
				
				// create Response object
				Response response = new Response(output);
				response.setRequest(request);
				response.sendStaticResource();
				
				// Close the socket
				socket.close();
				
				//check if the previous URI is a shutdown command
				shutdown = request.getUri().equals(SHUTDOWN_COMMAND);
			}
			catch (Exception e) {
				e.printStackTrace();
				continue;
			}
		}
	}
}
\end{lstlisting}
\subsection{Request}
Request表示一个HTTP请求,可以传递InputStream创建Request对象,并调用InputStream的read()获取原始数据。
\begin{lstlisting}
public class Request {
	
	private InputStream input;
	private String uri;
	
	public Request(InputStream input) {
		this.input = input;
	}
	//解析HTTP原始数据
	public void parse() {
		// Read a set of characters from the socket
		StringBuffer request = new StringBuffer(2048);
		int i;
		byte[] buffer = new byte[2048];
		try {
			i = input.read(buffer);
		}
		catch (IOException e) {
			e.printStackTrace();
			i = -1;
		}
		for (int j=0; j<i; j++) {
			request.append((char) buffer[j]);
		}
		System.out.print(request.toString());
		uri = parseUri(request.toString());
	}
	//解析HTTP的URI,按照消息头截取URI
	private String parseUri(String requestString) {
		int index1, index2;
		index1 = requestString.indexOf(' ');
		if (index1 != -1) {
			index2 = requestString.indexOf(' ', index1 + 1);
			if (index2 > index1)
			return requestString.substring(index1 + 1, index2);
		}
		return null;
	}
	
	public String getUri() {
		return uri;
	}
	
}
\end{lstlisting}
\subsection{Response}
\begin{lstlisting}
/*
HTTP Response = Status-Line
*(( general-header | response-header | entity-header ) CRLF)
CRLF
[ message-body ]
Status-Line = HTTP-Version SP Status-Code SP Reason-Phrase CRLF
*/

public class Response {
	
	private static final int BUFFER_SIZE = 1024;
	Request request;
	OutputStream output;
	
	public Response(OutputStream output) {
		this.output = output;
	}
	
	public void setRequest(Request request) {
		this.request = request;
	}
	//将静态资源作为原始数据发送
	public void sendStaticResource() throws IOException {
		byte[] bytes = new byte[BUFFER_SIZE];
		FileInputStream fis = null;
		try {
			File file = new File(HttpServer.WEB_ROOT, request.getUri());
			if (file.exists()) {
				fis = new FileInputStream(file);
				int ch = fis.read(bytes, 0, BUFFER_SIZE);
				while (ch!=-1) {
					output.write(bytes, 0, ch);
					ch = fis.read(bytes, 0, BUFFER_SIZE);
				}
			}
			else {
				// file not found
				String errorMessage = "HTTP/1.1 404 File Not Found\r\n" +
				"Content-Type: text/html\r\n" +
				"Content-Length: 23\r\n" +
				"\r\n" +
				"<h1>File Not Found</h1>";
				output.write(errorMessage.getBytes());
			}
		}
		catch (Exception e) {
			// thrown if cannot instantiate a File object
			System.out.println(e.toString() );
		}
		finally {
			if (fis!=null)
			fis.close();
		}
	}
}
\end{lstlisting}