\chapter{一个简单的servlet容器}
\section{javax.servlet.Servlet接口}
Servlet接口的5个方法:
\begin{lstlisting}
void init(ServletConfig var1) throws ServletException;
ServletConfig getServletConfig();
void service(ServletRequest var1, ServletResponse var2) throws ServletException, IOException;
String getServletInfo();
void destroy();
\end{lstlisting}
其中init()、service()和destroy()与生命周期有关,实例化servlet后会调用init()方法,只调用1次,然后执行service(),并且会被多次调用。
\par 在servlet移除前,会调用destroy(),清除自身持有的资源。
\section{应用程序1}
一个完整的servlet责任:
\begin{enumerate}
	\item 第一次调用,使用init()初始化;
	\item 针对每一个request请求,创建一个javax.serlvet.ServletRequest和javax.serlvet.ServletResponse实例;
	\item 调用service(),将ServletRequest和ServletResponse作为参数传入;
	\item 当关闭servlet,调用destroy(),卸载servlet。
\end{enumerate}
\subsection{HttpServer1类}
既可以处理静态资源请求,也可以处理servlet资源请求。
\begin{lstlisting}
public class HttpServer1 {
	
	/** WEB_ROOT is the directory where our HTML and other files reside.
	*  For this package, WEB_ROOT is the "webroot" directory under the working
	*  directory.
	*  The working directory is the location in the file system
	*  from where the java command was invoked.
	*/
	// shutdown command
	private static final String SHUTDOWN_COMMAND = "/SHUTDOWN";
	
	// the shutdown command received
	private boolean shutdown = false;
	
	public static void main(String[] args) {
		HttpServer1 server = new HttpServer1();
		server.await();
	}
	
	public void await() {
		ServerSocket serverSocket = null;
		int port = 8080;
		try {
			serverSocket =  new ServerSocket(port, 1, InetAddress.getByName("127.0.0.1"));
		}
		catch (IOException e) {
			e.printStackTrace();
			System.exit(1);
		}
		
		// Loop waiting for a request
		while (!shutdown) {
			Socket socket = null;
			InputStream input = null;
			OutputStream output = null;
			try {
				socket = serverSocket.accept();
				input = socket.getInputStream();
				output = socket.getOutputStream();
				
				// create Request object and parse
				Request request = new Request(input);
				request.parse();
				
				// create Response object
				Response response = new Response(output);
				response.setRequest(request);
				
				// check if this is a request for a servlet or a static resource
				// a request for a servlet begins with "/servlet/"
				if (request.getUri().startsWith("/servlet/")) {
					ServletProcessor1 processor = new ServletProcessor1();
					processor.process(request, response);
				}
				else {
					StaticResourceProcessor processor = new StaticResourceProcessor();
					processor.process(request, response);
				}
				
				// Close the socket
				socket.close();
				//check if the previous URI is a shutdown command
				shutdown = request.getUri().equals(SHUTDOWN_COMMAND);
			}
			catch (Exception e) {
				e.printStackTrace();
				System.exit(1);
			}
		}
	}
}
\end{lstlisting}
\subsection{Request类}
表示servlet的service()方法的参数request对象,这里只实现一部分。
\begin{lstlisting}
public class Request implements ServletRequest {
	
	private InputStream input;
	private String uri;
	
	public Request(InputStream input) {
		this.input = input;
	}
	
	public String getUri() {
		return uri;
	}
	
	private String parseUri(String requestString) {
		int index1, index2;
		index1 = requestString.indexOf(' ');
		if (index1 != -1) {
			index2 = requestString.indexOf(' ', index1 + 1);
			if (index2 > index1)
			return requestString.substring(index1 + 1, index2);
		}
		return null;
	}
	
	public void parse() {
		// Read a set of characters from the socket
		StringBuffer request = new StringBuffer(2048);
		int i;
		byte[] buffer = new byte[2048];
		try {
			i = input.read(buffer);
		}
		catch (IOException e) {
			e.printStackTrace();
			i = -1;
		}
		for (int j=0; j<i; j++) {
			request.append((char) buffer[j]);
		}
		System.out.print(request.toString());
		uri = parseUri(request.toString());
	}
	
	/* implementation of the ServletRequest*/
	public Object getAttribute(String attribute) {
		return null;
	}
	
	public Enumeration getAttributeNames() {
		return null;
	}
	
	public String getRealPath(String path) {
		return null;
	}
	
	public RequestDispatcher getRequestDispatcher(String path) {
		return null;
	}
	
	public boolean isSecure() {
		return false;
	}
	
	public String getCharacterEncoding() {
		return null;
	}
	
	public int getContentLength() {
		return 0;
	}
	
	public String getContentType() {
		return null;
	}
	
	public ServletInputStream getInputStream() throws IOException {
		return null;
	}
	
	public Locale getLocale() {
		return null;
	}
	
	public Enumeration getLocales() {
		return null;
	}
	
	public String getParameter(String name) {
		return null;
	}
	
	public Map getParameterMap() {
		return null;
	}
	
	public Enumeration getParameterNames() {
		return null;
	}
	
	public String[] getParameterValues(String parameter) {
		return null;
	}
	
	public String getProtocol() {
		return null;
	}
	
	public BufferedReader getReader() throws IOException {
		return null;
	}
	
	public String getRemoteAddr() {
		return null;
	}
	
	public String getRemoteHost() {
		return null;
	}
	
	public String getScheme() {
		return null;
	}
	
	public String getServerName() {
		return null;
	}
	
	public int getServerPort() {
		return 0;
	}
	
	public void removeAttribute(String attribute) {
	}
	
	public void setAttribute(String key, Object value) {
	}
	
	public void setCharacterEncoding(String encoding)
	throws UnsupportedEncodingException {
	}
}
\end{lstlisting}
\subsection{Response类}
与Request类似,这里只有部分实现,getWriter()实现。
\begin{lstlisting}
public class Response implements ServletResponse {
	
	private static final int BUFFER_SIZE = 1024;
	Request request;
	OutputStream output;
	PrintWriter writer;
	
	public Response(OutputStream output) {
		this.output = output;
	}
	
	public void setRequest(Request request) {
		this.request = request;
	}
	
	/* This method is used to serve a static page */
	public void sendStaticResource() throws IOException {
		byte[] bytes = new byte[BUFFER_SIZE];
		FileInputStream fis = null;
		try {
			/* request.getUri has been replaced by request.getRequestURI */
			File file = new File(Constants.WEB_ROOT, request.getUri());
			fis = new FileInputStream(file);
			/*
			HTTP Response = Status-Line
			*(( general-header | response-header | entity-header ) CRLF)
			CRLF
			[ message-body ]
			Status-Line = HTTP-Version SP Status-Code SP Reason-Phrase CRLF
			*/
			int ch = fis.read(bytes, 0, BUFFER_SIZE);
			while (ch!=-1) {
				output.write(bytes, 0, ch);
				ch = fis.read(bytes, 0, BUFFER_SIZE);
			}
		}
		catch (FileNotFoundException e) {
			String errorMessage = "HTTP/1.1 404 File Not Found\r\n" +
			"Content-Type: text/html\r\n" +
			"Content-Length: 23\r\n" +
			"\r\n" +
			"<h1>File Not Found</h1>";
			output.write(errorMessage.getBytes());
		}
		finally {
			if (fis!=null)
			fis.close();
		}
	}
	
	
	/** implementation of ServletResponse  */
	public void flushBuffer() throws IOException {
	}
	
	public int getBufferSize() {
		return 0;
	}
	
	public String getCharacterEncoding() {
		return null;
	}
	
	public Locale getLocale() {
		return null;
	}
	
	public ServletOutputStream getOutputStream() throws IOException {
		return null;
	}
	
	public PrintWriter getWriter() throws IOException {
		// autoflush is true, println() will flush,
		// but print() will not.
		writer = new PrintWriter(output, true);
		return writer;
	}
	
	public boolean isCommitted() {
		return false;
	}
	
	public void reset() {}
	public void resetBuffer() {}
	public void setBufferSize(int size) {}
	public void setContentLength(int length) {}
	public void setContentType(String type) {}
	public void setLocale(Locale locale) {}
}
\end{lstlisting}
如果PrintWriter()的第二个参数为真,对于println()的任何调用都会刷新输出,但是print()不会刷新输出,因此如果在service()中使用print()不会发送给浏览器。
\subsection{StaticResourceProcessor}
这里直接调用response.sendStaticResource()处理。
\begin{lstlisting}
public class StaticResourceProcessor {
	
	public void process(Request request, Response response) {
		try {
			response.sendStaticResource();
		}
		catch (IOException e) {
			e.printStackTrace();
		}
	}
}
\end{lstlisting}
\subsection{ServletProcessor1}
\begin{lstlisting}
public class ServletProcessor1 {
	
	public void process(Request request, Response response) {
		
		String uri = request.getUri();
		String servletName = uri.substring(uri.lastIndexOf("/") + 1);
		URLClassLoader loader = null;
		
		try {
			// create a URLClassLoader
			URL[] urls = new URL[1];
			URLStreamHandler streamHandler = null;
			File classPath = new File(Constants.WEB_ROOT);
			// the forming of repository is taken from the createClassLoader method in
			// org.apache.catalina.startup.ClassLoaderFactory
			String repository = (new URL("file", null, classPath.getCanonicalPath() + File.separator)).toString() ;
			// the code for forming the URL is taken from the addRepository method in
			// org.apache.catalina.loader.StandardClassLoader class.
			urls[0] = new URL(null, repository, streamHandler);
			loader = new URLClassLoader(urls);
		}
		catch (IOException e) {
			System.out.println(e.toString() );
		}
		Class myClass = null;
		try {
			myClass = loader.loadClass(servletName);
		}
		catch (ClassNotFoundException e) {
			System.out.println(e.toString());
		}
		
		Servlet servlet = null;
		
		try {
			servlet = (Servlet) myClass.newInstance();
			servlet.service((ServletRequest) request, (ServletResponse) response);
		}
		catch (Exception e) {
			System.out.println(e.toString());
		}
		catch (Throwable e) {
			System.out.println(e.toString());
		}
		
	}
}
\end{lstlisting}
URI的格式如:/serlvet/servletName,并通过字符串截取获取servletName;\\
为了载入类,创建一个类加载器java.net.URLClassLoader,是java.lang.ClassLoader直接子类,可以使用它的loadClass()载入servlet类;
\begin{lstlisting}
public URLClassLoader(URL[] urls)
\end{lstlisting}
当载入类时要指明在哪查找类,若以“/”结尾,则指向目录,否则指向一个JAR文件;这里把目录称为仓库(repository)。\\
这里使用下面URL构造函数:
\begin{lstlisting}
public URL(String protocol, String host, String file) throws MalformedURLException
//或
public URL(URL context, String spec, URLStreamHandler handler) throws MalformedURLException
\end{lstlisting}
如果只使用new URL(null, aString, null)不知道使用哪个构造函数,因此使用上述例子中URLStreamHandler区别构造函数。\\
加载类后,然后使用newInstance()创建新的实例,并调用它的服务程序service()。
\section{应用程序2}
\begin{lstlisting}
servlet = (Servlet) myClass.newInstance();
servlet.service((ServletRequest) request, (ServletResponse) response);
\end{lstlisting}
上述代码使用了向上转型,但是这是不安全的做法,因为了解此servlet容器内部结构的程序员,可以将ServletRequest和ServletResponse向下转型,就可以调用它们的公共方法。\\
不能将parse()和sendStaticResource()设置为private,但这两个方法在Servlet又应当是不可用,所以可以使用外观模式。
\par 添加两个外观类:RequestFacade和ResponseFacade,前者实现servletRequest接口,在其构造函数中需要指定一个Request对象传递给ServletRequest对象引用(private)。然后将外观类传递给service()方法,这里Servlet仍可以向下转型RequestFacade对象,但只能访问ServletRequest提供的方法。
\begin{lstlisting}
public class RequestFacade implements ServletRequest {
	
	private ServletRequest request = null;
	
	public RequestFacade(Request request) {
		this.request = request;
	}
	
	/* implementation of the ServletRequest*/
	public Object getAttribute(String attribute) {
		return request.getAttribute(attribute);
	}
	
	public Enumeration getAttributeNames() {
		return request.getAttributeNames();
	}
	//......	
\end{lstlisting}
对于servletProcessor2,改变部分:
\begin{lstlisting}
Servlet servlet = null;
RequestFacade requestFacade = new RequestFacade(request);
ResponseFacade responseFacade = new ResponseFacade(response);
try {
	servlet = (Servlet) myClass.newInstance();
	servlet.service((ServletRequest) requestFacade, (ServletResponse) responseFacade);
}
catch (Exception e) {
	System.out.println(e.toString());
}
\end{lstlisting}