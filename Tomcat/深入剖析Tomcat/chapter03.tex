\chapter{连接器}
Catalina两个主要板块:连接器(connector)和容器(container)。\\
连接器解析HTTP请求头,使servlet实例能够获取到请求头、cookie和请求参数/值等信息。
\section{StringManager}
Tomcat处理错误消息的方法是将错误信息存储在一个properties文件中,便于读取和编辑,但是随着类的暴增,错误信息的激增,不方便维护,因此将properties划分到不同的包中,会产生StringManager多个实例。可以采用单例模式减少资源的消耗,并且使用Hashtable保存该单个实例。
\section{应用程序}
3个模块:连接器模块、启动模块和核心模块。\par
启动模块:只有一个类(Bootstrap),负责启动应用程序。\par
连接器模块5类:
\begin{itemize}
	\item 连接器及其支持类(HttpConnector和HttpProcessor)
	\item 表示HTTP请求的类(HttpRequest)及其支持类
	\item 表示HTTP响应的类(HttpResponse)及其支持类
	\item 外观类(HttpRequestFacade和HTTPResponseFacade)
	\item 常量类
\end{itemize}
\par 核心模块:servletProcessor类和staticResourceProcessor。
\par 注:
\par HTTP请求对象使用HTTPRequest类表示,在调用servlet的service()时,HTTPRequest会被向上转型,因此必须正确地设置每一个HTTPRequest会用到的参数:URI、查询字符串、参数、Cookie和其他请求头。但连接器并不知道servlet会使用哪些,所以连接器必须解析所有从HTTP请求中获取的所有信息。
\par 解析HTTP请求设计开销大的字符串操作,因此可以让连接器只解析用到的数据。
\par Tomcat默认连接器和本节使用SocketInputStream类从套接字的InputStream中读取字节流。SocketInputStream的readRequestLine()会返回HTTP请求中第一行
内容(必须在readHeader()前调用),每回调用readHeader()会返回一个名/值对,可以重复调用直到获取所有请求头信息。两者返回值分别为HttpRequestLine实例和HttpHeader对象。
\par 等待HTTP请求由HTTPConnector完成,创建Request和Response由HttpProcessor完成。
HttpProcessor使用parse()解析HTTP请求行和请求头信息,填充到HttpRequest对象的成员变量中,但不负责解析请求体和查询字符串,这些由HttpRequest完成。
\subsection{启动应用程序}
\begin{lstlisting}
public final class Bootstrap {
	public static void main(String[] args) {
		HttpConnector connector = new HttpConnector();
		connector.start();
	}
}
\end{lstlisting}
实例化HttpConnector,并启动线程。
\begin{lstlisting}
public class HttpConnector implements Runnable {
	
	boolean stopped;
	private String scheme = "http";
	
	public String getScheme() {
		return scheme;
	}
	
	public void run() {
		ServerSocket serverSocket = null;
		int port = 8080;
		try {
			serverSocket =  new ServerSocket(port, 1, InetAddress.getByName("127.0.0.1"));
		}
		catch (IOException e) {
			e.printStackTrace();
			System.exit(1);
		}
		while (!stopped) {
			// Accept the next incoming connection from the server socket
			Socket socket = null;
			try {
				socket = serverSocket.accept();
			}
			catch (Exception e) {
				continue;
			}
			// Hand this socket off to an HttpProcessor
			HttpProcessor processor = new HttpProcessor(this);
			processor.process(socket);
		}
	}
	
	public void start() {
		Thread thread = new Thread(this);
		thread.start();
	}
}
\end{lstlisting}
\subsection{HttpConnector类}
HttpConnector实现了java.lang.Runnable接口。
\par HttpProcessor类的process()接受传入的HTTP请求的套接字,然后完成下面4个操作:
\begin{itemize}
	\item 创建一个HttpRequest对象;
	\item 创建一个HttpResponse对象;
	\item 解析HTTP请求的第一行内容的请求头信息,填充HttpRequest对象;
	\item 将HttpRequest对象和HttpResponse对象传递给servletProcessor或
	StaticResourceProcessor的process()。
\end{itemize}
HttpProcessor的process()方法:
\begin{lstlisting}
public void process(Socket socket) {
	SocketInputStream input = null;
	OutputStream output = null;
	try {
		input = new SocketInputStream(socket.getInputStream(), 2048);
		output = socket.getOutputStream();
		
		// create HttpRequest object and parse
		request = new HttpRequest(input);
		
		// create HttpResponse object
		response = new HttpResponse(output);
		response.setRequest(request);
		
		response.setHeader("Server", "Pyrmont Servlet Container");
		
		parseRequest(input, output);
		parseHeaders(input);
		
		//check if this is a request for a servlet or a static resource
		//a request for a servlet begins with "/servlet/"
		if (request.getRequestURI().startsWith("/servlet/")) {
			ServletProcessor processor = new ServletProcessor();
			processor.process(request, response);
		}
		else {
			StaticResourceProcessor processor = new StaticResourceProcessor();
			processor.process(request, response);
		}
		
		// Close the socket
		socket.close();
		// no shutdown for this application
	}
	catch (Exception e) {
		e.printStackTrace();
	}
}
\end{lstlisting}
\par 先获取套接字的输入输出流,然后创建一个HttpRequest实例和一个HttpResponse实例,然后HttpRequest实例赋值给HTTPResponse实例;
\par 发送响应头信息,然后会调用setHeader()设置头信息,再调用私有方法解析请求,
可以根据URI模式判断是Servlet还是静态资源,最后关闭套接字。
\subsection{创建HttpRequest对象}
HttpRequest的请求头、Cookie和请求参数数据:
\begin{lstlisting}
protected HashMap headers = new HashMap();
protected ArrayList cookies = new ArrayList();
protected ParameterMap parameters = null;
\end{lstlisting}
\begin{enumerate}
	\item 读取套接字的输入流\\
	可以按照前面的java.io.InputStream类的read()方法进行获取方法、URI、HTTP协议的版本信息。
	\begin{lstlisting}
byte[] buffer = new byte[2048];
try {
	i = input.read(buffer);
}
	\end{lstlisting}
	这里使用SocketInputStream进行解析,使用它是为了调用其方法readRequestLine()和readHeader():
	\begin{lstlisting}
SocketInputStream input = null;
OutputStream output = null;
try {
	input = new SocketInputStream(socket.getInputStream(), 2048);
	//......
}
	\end{lstlisting}
	\item 解析请求行\\
	HttpProcessor的process()会调用私有方法parseRequest()解析请求行,
	请求行的URI可以包含0或多个请求参数。\\
	在servlet/JSP中,参数名jsessionid用于携带一个会话标识符,会话标识符通常是作为Cookie嵌入的,但当浏览器禁用Cookie时,可以将它嵌入到查询符中。
	\begin{lstlisting}
 private void parseRequest(SocketInputStream input, OutputStream output)
throws IOException, ServletException {
	
	// Parse the incoming request line
	input.readRequestLine(requestLine);
	String method =
	new String(requestLine.method, 0, requestLine.methodEnd);
	String uri = null;
	String protocol = new String(requestLine.protocol, 0, requestLine.protocolEnd);
	
	// Validate the incoming request line
	if (method.length() < 1) {
		throw new ServletException("Missing HTTP request method");
	}
	else if (requestLine.uriEnd < 1) {
		throw new ServletException("Missing HTTP request URI");
	}
	// Parse any query parameters out of the request URI
	int question = requestLine.indexOf("?");
	if (question >= 0) {
		request.setQueryString(new String(requestLine.uri, question + 1,
		requestLine.uriEnd - question - 1));
		uri = new String(requestLine.uri, 0, question);
	}
	else {
		request.setQueryString(null);
		uri = new String(requestLine.uri, 0, requestLine.uriEnd);
	}
	
	
	// Checking for an absolute URI (with the HTTP protocol)
	if (!uri.startsWith("/")) {
		int pos = uri.indexOf("://");
		// Parsing out protocol and host name
		if (pos != -1) {
			pos = uri.indexOf('/', pos + 3);
			if (pos == -1) {
				uri = "";
			}
			else {
				uri = uri.substring(pos);
			}
		}
	}
	
	// Parse any requested session ID out of the request URI
	String match = ";jsessionid=";
	int semicolon = uri.indexOf(match);
	if (semicolon >= 0) {
		String rest = uri.substring(semicolon + match.length());
		int semicolon2 = rest.indexOf(';');
		if (semicolon2 >= 0) {
			request.setRequestedSessionId(rest.substring(0, semicolon2));
			rest = rest.substring(semicolon2);
		}
		else {
			request.setRequestedSessionId(rest);
			rest = "";
		}
		request.setRequestedSessionURL(true);
		uri = uri.substring(0, semicolon) + rest;
	}
	else {
		request.setRequestedSessionId(null);
		request.setRequestedSessionURL(false);
	}
	
	// Normalize URI (using String operations at the moment)
	String normalizedUri = normalize(uri);
	
	// Set the corresponding request properties
	((HttpRequest) request).setMethod(method);
	request.setProtocol(protocol);
	if (normalizedUri != null) {
		((HttpRequest) request).setRequestURI(normalizedUri);
	}
	else {
		((HttpRequest) request).setRequestURI(uri);
	}
	
	if (normalizedUri == null) {
		throw new ServletException("Invalid URI: " + uri + "'");
	}
}
\end{lstlisting}
首先会调用SocketInputStream类的readRequestLine()方法,填充HttpRequestLine实例;\\
然后判断URI是否包含参数,如果包含参数,对参数进行处理,否则直接当做url处理。\\
查询字符串可能会包含一个会话标识符jsessionid,若存在参数jsessionid,则表明会话标识符在查询字符串中,而不再Cookie中。\\
然后,会将URI传入到normalize()方法中,对非正常URL进行修正。
\item 解析请求头\\
\begin{itemize}
	\item 可以通过其类的无参构造函数创建一个HttpHeader实例;
	\item 创建了HttpHeader实例后,可以将其传递给SocketInputStream的readHeader()方法,然后会填充HttpHeader对象。
	\item 要获取请求头的名字和值,可以使用
	\begin{lstlisting}
String name = new String(header.name, 0, header.nameEnd);
String value = new String(header.value, 0, header.valueEnd);
	\end{lstlisting}
	\item parseHeaders()会有一个while循环,直到读完为止。
	\item 另外,一些请求头包含属性设置信息,如“content-length”和“cookies”
\end{itemize}
\item 解析Cookie\\
Cookie是由浏览器作为HTTP请求头的一部分发送的,对Cookie的解析可以通过org.apache.catalina.util.RequestUtil类的parseCookieHeader()方法完成。
\begin{lstlisting}
public static Cookie[] parseCookieHeader(String header) {
	
	if ((header == null) || (header.length() < 1))
	return (new Cookie[0]);
	
	ArrayList cookies = new ArrayList();
	while (header.length() > 0) {
		int semicolon = header.indexOf(';');
		if (semicolon < 0)
		semicolon = header.length();
		if (semicolon == 0)
		break;
		String token = header.substring(0, semicolon);
		if (semicolon < header.length())
		header = header.substring(semicolon + 1);
		else
		header = "";
		try {
			int equals = token.indexOf('=');
			if (equals > 0) {
				String name = token.substring(0, equals).trim();
				String value = token.substring(equals+1).trim();
				cookies.add(new Cookie(name, value));
			}
		} catch (Throwable e) {
			;
		}
	}
	
	return ((Cookie[]) cookies.toArray(new Cookie[cookies.size()]));
	
}
\end{lstlisting}
\item 获取参数
在调用javax.servlet.http.HttpservletRequest的getParameter()、
getParameterMap()、getParameterNames()或getParameterValues()方法
之前,都不需要解析查询字符串或HTTP请求体来获得参数,因此在HttpRequest类中,这4个方法实现都会先调用parseParameter()方法。\\
参数只解析一次,HTTPRequest中使用一个parsed判断是否完成解析。\\
参数可以出现在查询字符串或请求体中,若使用GET方法请求,所有参数在查询字符串中,若使用POST,则请求体中也可能会有参数。\\
这里使用java.util.HashMap的子类ParameterMap存放参数,有一个locked判断是否可以修改。
\begin{lstlisting}
public final class ParameterMap extends HashMap {
	
	public ParameterMap() {
		super();
	}
	
	public ParameterMap(int initialCapacity) {
		super(initialCapacity);
	}

	public ParameterMap(int initialCapacity, float loadFactor) {
		super(initialCapacity, loadFactor);
	}
	
	public ParameterMap(Map map) {
		super(map);
	}
	
	private boolean locked = false;

	public boolean isLocked() {
		
		return (this.locked);
		
	}

	public void setLocked(boolean locked) {	
		this.locked = locked;
	}


	public void clear() {
		if (locked)
		throw new IllegalStateException
		(sm.getString("parameterMap.locked"));
		super.clear();
	}

	public Object put(Object key, Object value) {
		if (locked)
		throw new IllegalStateException
		(sm.getString("parameterMap.locked"));
		return (super.put(key, value));
	}

	public void putAll(Map map) {
		if (locked)
		throw new IllegalStateException
		(sm.getString("parameterMap.locked"));
		super.putAll(map);
	}

	public Object remove(Object key) {
		if (locked)
		throw new IllegalStateException
		(sm.getString("parameterMap.locked"));
		return (super.remove(key));
	}
}
\end{lstlisting}
由于参数可以在查询字符串或HTTP请求体中,因此parseParameters()必须对二者检查,当解析完成时,参数会存储到对象变量parameters中,下面是解析过程代码:
\begin{lstlisting}
protected void parseParameters() {
	if (parsed)
		return;
	ParameterMap results = parameters;
	if (results == null)
		results = new ParameterMap();
	results.setLocked(false);
	String encoding = getCharacterEncoding();
	if (encoding == null)
		encoding = "ISO-8859-1";
	
	// Parse any parameters specified in the query string
	String queryString = getQueryString();
	try {
		RequestUtil.parseParameters(results, queryString, encoding);
	}
	catch (UnsupportedEncodingException e) {
		;
	}
	
	// Parse any parameters specified in the input stream
	String contentType = getContentType();
	if (contentType == null)
		contentType = "";
	int semicolon = contentType.indexOf(';');
	if (semicolon >= 0) {
		contentType = contentType.substring(0, semicolon).trim();
	}
	else {
		contentType = contentType.trim();
	}
	if ("POST".equals(getMethod()) && (getContentLength() > 0)
	&& "application/x-www-form-urlencoded".equals(contentType)) {
		try {
			int max = getContentLength();
			int len = 0;
			byte buf[] = new byte[getContentLength()];
			ServletInputStream is = getInputStream();
			while (len < max) {
				int next = is.read(buf, len, max - len);
				if (next < 0 ) {
					break;
				}
				len += next;
			}
			is.close();
			if (len < max) {
				throw new RuntimeException("Content length mismatch");
			}
			RequestUtil.parseParameters(results, buf, encoding);
		}
		catch (UnsupportedEncodingException ue) {
			;
		}
		catch (IOException e) {
			throw new RuntimeException("Content read fail");
		}
	}
	
	// Store the final results
	results.setLocked(true);
	parsed = true;
	parameters = results;
}
\end{lstlisting}
首先会判断是否解析过参数,如果没有,则进行解析;\\
打开parameterMap的锁,再检查字符串encoding,若为null,则设置为"ISO-8859-1",然后使用org.apache.Catalina.util.RequestUtil的parseParameters()完成。\\
然后,parseParameters()会检查HTTP请求体是否包含参数,若使用POST,请求体包含参数,且“content-length”的值会大于0,“content-type”的值为“pplication/x-www-form-urlencoded”。
\end{enumerate}
\subsection{创建HttpResponse对象}
这里为了解决print()和println()不会自动将结果发送到客户端,使用一个新的
writer来flush()数据,这里可以使用ResponseStream的帝乡来实例化ResponseWriter,(实际上使用了java.io.OutputStreamWriter)告诉writer输出流,并实现自动发送结果。\\
使用OutputStreamWriter会将传入的字符转换为使用指定字符集的字节数组。
\begin{lstlisting}
 public PrintWriter getWriter() throws IOException {
	ResponseStream newStream = new ResponseStream(this);
	newStream.setCommit(false);
	OutputStreamWriter osr =
		new OutputStreamWriter(newStream, getCharacterEncoding());
	writer = new ResponseWriter(osr);
	return writer;
}
\end{lstlisting}
\subsection{静态资源处理器和servlet处理器}
与之前的处理servlet的方法类似,只有一个方法:process(),但是不同的是,这里的参数是HttpRequest和HttpResponse,而不是Request和Response,并且这里使用外观类作为调用后,使用finshResponse()处理连接。
\begin{lstlisting}
servlet = (Servlet) myClass.newInstance();
HttpRequestFacade requestFacade = new HttpRequestFacade(request);
HttpResponseFacade responseFacade = new HttpResponseFacade(response);
servlet.service(requestFacade, responseFacade);
((HttpResponse) response).finishResponse();
\end{lstlisting}
