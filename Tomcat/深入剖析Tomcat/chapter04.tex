\chapter{Tomcat的默认连接器}
Tomcat的连接器有默认连接器,还有Coyote、mod\_jk和mod\_webapp等,
并必须满足下面要求:
\begin{itemize}
	\item 实现org.apache.catalina.Connector接口;
	\item 负责创建实现了org.apache.catalina.Request接口的request对象;
	\item 负责创建实现了org.apache.catalina.Response接口的response对象。
\end{itemize}
\par 默认连接器原理:它会等待引入的HTTP请求,创建request对象和response对象,然后调用org.apache.catalina.Container接口的invoke()方法,将request对象和response对象传给servlet容器。
\par 相对之前的连接器,使用优化方法:使用一个对象池避免了频繁的创建对象带来的性能损耗,其次,使用字符数组代替字符串。
\par Tomcat的默认连接器支持HTTP1.1以及以前版本。
\section{HTTP1.1的新特性}
\subsection{持久连接}
HTTP1.1之前,无论浏览器何时连接服务器,当服务器处理请求后,就会断开与浏览器的连接,但是页面的引用资源就需要另开新连接获取,因此处理过程很慢。
\par 使用持久连接,下载页面后,服务器不会关闭连接,等待客户端请求引用资源,HTTP 1.1默认使用持久连接,可以显式使用:
\par connection: keep-alive
\subsection{块编码}
建立持久连接后,服务器可以从多个资源发送字节流,客户端也可以使用该连接
发送多个请求,因此,发送方必须在每个请求或响应加上“content-length”头信息,这样接收方才能解释接受到的字节信息,但发送方有时不知道发送多少个字节,servlet容器可以在接受一部分信息就能响应消息,因此可以使得接收方不知道发送内容长度也可以解析内容。
\par 在HTTP 1.0中,服务器可以不写“content-length”头信息,尽管往连接中写响应信息即可,发送完信息后,直接关闭连接,而客户端会一直读取内容,直到读方法返回-1。
\par HTTP 1.1使用“transfer-encoding”的特殊请求头指明字节流将会分块发送。对每个块,块的长度(16进制)会有一个回车,然后是回车/换行符(CR/LF),然后是具体的数据,一个事务以一个长度为0的块标记,表示事务完成。
\subsection{状态码100的使用}
使用HTTP 1.1的客户端可以向服务器发送请求体之前发送如下请求头,并等待服务器的确认:
Except: 100-continue
\par 当客户端发送一个较长请求体之前,不确定服务端是否会接受,就可能使用上述头信息,服务器发送下面表明接收:
HTTP/1.1 100 Continue。
\par 注:返回内容后面要加上CRLF字符,然后服务器继续读取输入流信息。

\section{Connector接口}
Tomcat连接器需要实现org.apache.catalina.Connector接口,并实现
getContainer()、setContainer()、createRequest()和createResponse()方法。
\par setContainer()将连接器与某个servlet容器相关联;getContainer()返回当前连接器的servlet容器;createRequest()为引入的HTTP请求创建request对象;createResponse()创建一个response对象。
\section{HttpConnector类}
\subsection{创建服务器套接字}
HttpConnector类的initialize()方法会调用私有办法open(),后者返回一个
ServerSocket实例,赋值给成员变量serverSocket,但没有使用它的构造函数,
而是使用工厂ServerSocketFactory获取实例。
\subsection{维护HttpProcessor实例}
Tomcat的默认连接器中,HttpConnector实例有一个HttpProcessor对象池,
每个HttpProcessor实例都运行在自己的线程中,以应对对个HTTP请求。
\par HTTPProcessor对象池:
\begin{lstlisting}
private Stack processors = new Stack();
protected int minProcessors = 5;
private int maxProcessors = 20;
\end{lstlisting}
\par 创建的HttpProcessor实例数目由minProcessors和maxProcessors决定,默认min=5,max=20。
初始时,会创建min个实例等待,当请求数目超过时,会创建更多的HttpProcessor实例,直到max,如果还不够,则忽略请求。
\par 如果希望HttpConnector能够持续创建HttpProcessor实例,可以将max设置为负数,且curProcessor保存了当前HttpProcessor实例的数目。
\par HttpConnector的start()创建初始数量HttpProcessor实例:
\begin{lstlisting}
// Create the specified minimum number of processors
while (curProcessors < minProcessors) {
	if ((maxProcessors > 0) && (curProcessors >= maxProcessors))
		break;
	HttpProcessor processor = newProcessor();
	recycle(processor);
}
\end{lstlisting}
\par HttpProcessor负责解析HTTP的请求行和请求头,它的构造函数会调用HttpConnector的createRequest和createResponse方法。
\subsection{提供HTTP请求服务}
对于每个引入的HTTP请求,通过调用私有方法createProcessor()获得一个HttpProcessor对象,可以从对象池中获取,如果对象池中有可用,则直接返回,否则判断min和max大小进行进一步生成。
\begin{lstlisting}
if (processor == null) {
	try {
		log(sm.getString("httpConnector.noProcessor"));
		socket.close();
	} catch (IOException e) {
		;
	}
	continue;
}
processor.assign(socket);
\end{lstlisting}
如果为空,说明连接已达上限,直接关闭socket,否则设置processor的socket。
\section{HttpProcessor类}
HttpProcessor的process()负责解析HTTP请求。
\par 之前的HttpConnector只能等待HttpProcessor处理完毕后才能处理下一个
请求(同步),因此,在默认连接器中,HttpProcessor实现了java.lang.Runnable接口,可以在自己的“处理器线程”处理请求,而不阻塞HttpConnector线程。
\par run()方法中while执行到await()方法时阻塞,await()会阻塞处理器,
直到它从HttpConnector中获取到了新的Socket对象,在调用assign之前一直阻塞,但assign并不是与await在同一个线程中。
\begin{lstlisting}
synchronized void assign(Socket socket) {
	
	// Wait for the Processor to get the previous Socket
	while (available) {
		try {
			wait();
		} catch (InterruptedException e) {
		}
	}
	
	// Store the newly available Socket and notify our thread
	this.socket = socket;
	available = true;
	notifyAll();
	
	if ((debug >= 1) && (socket != null))
	log(" An incoming request is being assigned");
	
}
private synchronized Socket await() {
	
	// Wait for the Connector to provide a new Socket
	while (!available) {
		try {
			wait();
		} catch (InterruptedException e) {
		}
	}
	
	// Notify the Connector that we have received this Socket
	Socket socket = this.socket;
	available = false;
	notifyAll();
	
	if ((debug >= 1) && (socket != null))
	log("  The incoming request has been awaited");
	
	return (socket);
	
}
\end{lstlisting}
处理器线程刚启动,available变为false,会等待其他线程调用notify()或notifyAll()方法,直到连接器线程调用了实例HttpProcessor的notifyAll()方法。
\par 当新的Socket对象通过assign()传入时,连接器线程会调用HttpProcessor实例的assign()方法,由于变量available的值为false,因此会调出while,传入的Socket会赋值给HttpProcessor的socket变量。
\par 然后连接器线程会将available设置为true,并调用notifyAll(),唤醒处理器线程,这样处理器线程会跳出while循环,将成员变量赋值给一个局部变量,然后调用notifyAll(),将局部变量返回,由其他程序继续处理。
\par 注:
\par 为什么await()方法使用局部变量socket,而不直接返回?因为使用局部变量可以在当前Socket对象处理完之前,继续接收下一个Socket对象。
\par 为什么await()需要调用notifyAll()方法?是为了防止出现另一个Socket对象已经到达,而此时变量available还是true。在这种情况下,“连接器线程”会在assign()方法内的循环中阻塞,直到“处理器线程”调用notifyAll()方法。
\section{Request对象}
RequestBase类直接实现了org.apache.catalina.Request,而RequestBase类是HttpRequest类的父类,最终实现类是HttpRequestImpl类,而HttpRequestImpl类继承自HttpRequest类。
\section{Response对象}
里面的继承关系和Request类似。
\section{处理请求}
当Socket对象被赋值给HttpProcessor后,HttpProcessor的run()会调用process()方法,process()方法会执行3个操作:解析连接、解析请求、解析请求头。
\par process()使用ok表示处理过程是否有错误发生,finishResponse表示是否调用Response接口的finishResponse()方法,keepAlive是否为持久连接,stopped表示是否被连接器终止,http11表示是否支持HTTP1.1客户端。
\par 在process()中,先执行request和response的初始化操作,然后调用
parseConnection()、parseRequest()和parseHeader()方法解析HTTP请求,
parseConnection()获取请求使用的协议,并根据协议版本设置是否持久连接,
如果在http请求头中发现“Except: 100-continue”,则parseHeader()将sendAck设置为true。
\par 若请求协议HTTP 1.1,而且客户端发出了“Except: 100-continue”,会对调用
ackRequest()请求该方法进行响应,另外还会检查是否允许分块。
\begin{lstlisting}
//响应信息
HTTP/1.1 100 Continue
\end{lstlisting}
\par 完成解析后,process()将request和response对象作为参数传入servlet容器的invoke()方法。
\begin{lstlisting}
try {
	((HttpServletResponse) response).setHeader
	("Date", FastHttpDateFormat.getCurrentDate());
	if (ok) {
		connector.getContainer().invoke(request, response);
	}
}
\end{lstlisting}
\par 然后,如果finishResponse为true,则调用response对象的finishResponse()方法和request对象的finishRequest()方法,将结果发送至客户端。
最后,检查response的头信息“Connection”是否在servlet中设置为close或者协议为1.0,如果任一种为真,则将keepAlive设置为false,最后对对象request和response进行回收。
\par keepAlive为true,或前面解析没有错误,或HttpProcessor没有被终止,
则while将继续运行,否则调用shutdownInput(),并关闭套接字。
\subsection{解析连接}
parseConnection()会从套接字中获取Internet地址,将其赋值给HttpRequestImpl对象,并且会检查是否使用了代理,将Socket对象赋值给request对象。
\begin{lstlisting}
private void parseConnection(Socket socket)
throws IOException, ServletException {
	if (debug >= 2)
	log("  parseConnection: address=" + socket.getInetAddress() +
	", port=" + connector.getPort());
	((HttpRequestImpl) request).setInet(socket.getInetAddress());
	if (proxyPort != 0)
	request.setServerPort(proxyPort);
	else
	request.setServerPort(serverPort);
	request.setSocket(socket);
}
\end{lstlisting}
\subsection{解析请求}
parseRequest()与之前版本类似。
\subsection{解析请求头}
默认连接器中的parseHeaders()方法使用org.apache.catalina.http包中HttpHeader类和DefaultHeader类。
\par parseHeaders()使用while读取所有HTTP的请求信息,while循环调用request
对象的allocateHeader()获取一个内容为空的HttpHeader实例,然后,该实例会被传入SocketInputStream的readHeader()方法中。
\par 若所有请求头都已经读取过了,则readHeader()不会再给HttpHeader实例设置name属性,这时就可以退出parseHeader()方法。
\par 如果一个HttpHeader有name,那么它肯定有value:
\begin{lstlisting}
private void parseHeaders(SocketInputStream input) throws IOException, ServletException {
	while (true) {
		HttpHeader header = request.allocateHeader();
		
		// Read the next header
		input.readHeader(header);
		if (header.nameEnd == 0) {
			if (header.valueEnd == 0) {
				return;
			} else {
				throw new ServletException
				(sm.getString("httpProcessor.parseHeaders.colon"));
			}
		}
		
		String value = new String(header.value, 0, header.valueEnd);		
\end{lstlisting}
\par 然后,将读取到的请求头的name属性值与DefaultHeaders中的标准名称
比较,这里比较的是字符数组而不是字符串。
\begin{lstlisting}
// Set the corresponding request headers
if (header.equals(DefaultHeaders.AUTHORIZATION_NAME)) {
	request.setAuthorization(value);
} else if (header.equals(DefaultHeaders.ACCEPT_LANGUAGE_NAME)) {
	parseAcceptLanguage(value);
} else if (header.equals(DefaultHeaders.COOKIE_NAME)) {
	//parse cookie
} else if (header.equals(DefaultHeaders.CONTENT_LENGTH_NAME)) {
	//get content legth
} else if (header.equals(DefaultHeaders.CONTENT_TYPE_NAME)) {
	request.setContentType(value);
} else if (header.equals(DefaultHeaders.HOST_NAME)) {
	//get host name
} else if (header.equals(DefaultHeaders.CONNECTION_NAME)) {
	if (header.valueEquals
	(DefaultHeaders.CONNECTION_CLOSE_VALUE)) {
		keepAlive = false;
		response.setHeader("Connection", "close");
	}
//request.setConnection(header);
/*
if ("keep-alive".equalsIgnoreCase(value)) {
keepAlive = true;
}
*/
} else if (header.equals(DefaultHeaders.EXPECT_NAME)) {
	if (header.valueEquals(DefaultHeaders.EXPECT_100_VALUE))
		sendAck = true;
	else
		throw new ServletException (sm.getString ("httpProcessor.parseHeaders.unknownExpectation"));
} else if (header.equals(DefaultHeaders.TRANSFER_ENCODING_NAME)) {
	//request.setTransferEncoding(header);
}

request.nextHeader();
\end{lstlisting}
\section{简单的Container应用程序}
SimpleContainer实现了org.catalina.Container接口,它可以与默认连接器进行关联,Bootstrap用于启动该应用程序,但移除了连接器模块,以及ServletProcessor和StaticResourceProcessor的使用。
\begin{lstlisting}
public class SimpleContainer implements Container {
	
	public static final String WEB_ROOT =
	System.getProperty("user.dir") + File.separator  + "webroot";
	
	public SimpleContainer() {
	}
	
	public String getInfo() {
		return null;
	}
	
	public Loader getLoader() {
		return null;
	}
	
	public void setLoader(Loader loader) {
	}
	
	public Logger getLogger() {
		return null;
	}
	
	public void setLogger(Logger logger) {
	}
	
	public Manager getManager() {
		return null;
	}
	
	public void setManager(Manager manager) {
	}
	
	public Cluster getCluster() {
		return null;
	}
	
	public void setCluster(Cluster cluster) {
	}
	
	public String getName() {
		return null;
	}
	
	public void setName(String name) {
	}
	
	public Container getParent() {
		return null;
	}
	
	public void setParent(Container container) {
	}
	
	public ClassLoader getParentClassLoader() {
		return null;
	}
	
	public void setParentClassLoader(ClassLoader parent) {
	}
	
	public Realm getRealm() {
		return null;
	}
	
	public void setRealm(Realm realm) {
	}
	
	public DirContext getResources() {
		return null;
	}
	
	public void setResources(DirContext resources) {
	}
	
	public void addChild(Container child) {
	}
	
	public void addContainerListener(ContainerListener listener) {
	}
	
	public void addMapper(Mapper mapper) {
	}
	
	public void addPropertyChangeListener(PropertyChangeListener listener) {
	}
	
	public Container findChild(String name) {
		return null;
	}
	
	public Container[] findChildren() {
		return null;
	}
	
	public ContainerListener[] findContainerListeners() {
		return null;
	}
	
	public Mapper findMapper(String protocol) {
		return null;
	}
	
	public Mapper[] findMappers() {
		return null;
	}
	
	public void invoke(Request request, Response response)
	throws IOException, ServletException {
		
		String servletName = ( (HttpServletRequest) request).getRequestURI();
		servletName = servletName.substring(servletName.lastIndexOf("/") + 1);
		URLClassLoader loader = null;
		try {
			URL[] urls = new URL[1];
			URLStreamHandler streamHandler = null;
			File classPath = new File(WEB_ROOT);
			String repository = (new URL("file", null, classPath.getCanonicalPath() + File.separator)).toString() ;
			urls[0] = new URL(null, repository, streamHandler);
			loader = new URLClassLoader(urls);
		}
		catch (IOException e) {
			System.out.println(e.toString() );
		}
		Class myClass = null;
		try {
			myClass = loader.loadClass(servletName);
		}
		catch (ClassNotFoundException e) {
			System.out.println(e.toString());
		}
		
		Servlet servlet = null;
		
		try {
			servlet = (Servlet) myClass.newInstance();
			servlet.service((HttpServletRequest) request, (HttpServletResponse) response);
		}
		catch (Exception e) {
			System.out.println(e.toString());
		}
		catch (Throwable e) {
			System.out.println(e.toString());
		}
		
		
		
	}
	
	public Container map(Request request, boolean update) {
		return null;
	}
	
	public void removeChild(Container child) {
	}
	
	public void removeContainerListener(ContainerListener listener) {
	}
	
	public void removeMapper(Mapper mapper) {
	}
	
	public void removePropertyChangeListener(PropertyChangeListener listener) {
	}
	
}
\end{lstlisting}
\par 默认连接器会调用这里的invoke(),invoke()会创建一个类载入器,载入相关的
servlet类,并调用servlet类的service()方法,该方法与之前的ServletProcessor的process()类似。
\begin{lstlisting}
public final class Bootstrap {
	public static void main(String[] args) {
		HttpConnector connector = new HttpConnector();
		SimpleContainer container = new SimpleContainer();
		connector.setContainer(container);
		try {
			connector.initialize();
			connector.start();
			
			// make the application wait until we press any key.
			System.in.read();
		}
		catch (Exception e) {
			e.printStackTrace();
		}
	}
}
\end{lstlisting}
\par Bootstrap类的main()方法分别创建了org.apache.catalina.connector.http.HttpConnector类和SimpleContainer的一个实例,然后调用连接器的setContainer()将连接器和servlet容器相关联,接下来,调用连接器的initialize()和start()方法。