\chapter{servlet容器}
servlet容器是用来处理请求servlet资源,并为Web客户端填充response对象的模块。
Tomcat中有4种容器:Engine、Host、Context和Wrapper。
\section{Container接口}
Tomcat的servlet容器必须实现org.apache.catalina.Container接口,需要将servlet容器的实例作为参数传入连接器的setContainer()方法中,以此使用invoke()。
\par 4种容器:
\begin{itemize}
	\item Engine:表示整个Catania servlet引擎;
	\item Host:表示包含一个或多个Context容器的虚拟主机;
	\item Context:表示一个Web应用程序,一个Context可以有多个Wrapper;
	\item Wrapper:表示一个独立的servlet。
\end{itemize}
\par 4个接口的标准实现分别是StandardEngine类、StandardHost类、StandardContext类和StandardWrapper类,均在org.apache.catalina.core包内。
\par 一个容器可以有0个或多个低层级的子容器。一般情况下,一个Context会有
一个或多个Wrapper实例,一个Host实例会有0个或多个Context实例,Wrapper处于最底层,不在包含子容器。
\par 可以调用Container接口的addChild()方法向某容器中添加子容器。
\par public void addChild(Container child);
\par Container接口提供findChild()方法和findChildren()方法来查找子容器和所有子容器的某个容器。
\begin{lstlisting}
public Container findChild(String name);
public Container[] findChildren();
\end{lstlisting}
\par Container接口的设计满足:在部署应用时,Tomcat可以通过编辑配置文件
(server.xml)来决定使用哪种容器。

\section{管道任务}
相关接口:Pipeline、Value、ValueContext和Contained。
\par 管道包含该servlet将要调用的任务,一个阀表示一个具体的执行任务。
\par 在servlet容器中,有一个基础阀,可以添加任意数量阀,阀的数量指的是额外添加阀数量,基础阀总是最后一个执行。
\par 一个servlet容器可以有一条管道,当调用了容器的invoke(),容器会将处理工作交由阀继续执行任务,直到所有的阀都处理完成。