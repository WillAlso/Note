\chapter{对象入门}
面向对象编程需要使用形象思维,而不是程序化思维。
\section{抽象的进步}
所有编程语言的最终目的都是提供一种“抽象”方法。\\
不能将所有问题归结为算法(LISP),也不能归结为决策链(PROLOG)。\\
面向对象编程,允许根据问题描述问题。面向对象:\\
\begin{enumerate}
	\item 所有东西都是对象;
	\item 程序是一大堆对象的组合;
	\item 每个对象都有自己的存储空间,可以容纳其他对象;
	\item 每个对象都有一种类型;
	\item 同一类所有对象都能接受相同的消息。
\end{enumerate}

\section{对象的接口}
每个对象隶属于一个特定的“类”,那个类具有自己得通用特征与行为。\\
每个对象仅能接受特定的请求,发出的请求是通过接口定义的,对象的“类型”或“类”规定了接口形式。\\
\section{实现方案的隐藏}
面向对象编程:类创建者、客户程序员,客户程序员最主要的目标是收集一个充斥着各种类的编程工具箱,以便快速开发自己的应用;创建者需要隐藏类的内部。\\
接口规定了可对一个特定的对象发出哪些请求。