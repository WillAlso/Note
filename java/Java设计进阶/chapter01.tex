\chapter{线程 上}
课前问题:
\begin{enumerate}
	\item 程序代码一般都是顺序执行的,如何让程序在运行过程中,实现
	多个代码段的同时运行?
	\item 如何创建多个线程?创建线程的两种方式是什么?
	\item 线程为什么要休眠?
	\item Thread类都有哪些方法?
\end{enumerate}
\section{线程的基本概念}
\noindent 线程:一个线程是一个程序内部的顺序控制流。
\\线程和进程
\begin{itemize}
	\item 每个进程都有独立的代码和数据空间(进程上下文),进程切换开销大。
	\item 线程是轻量的进程,同一类线程共享代码和数据空间,每个线程有
	独立的运行栈和程序计数器(PC),线程切换的开销小。
\end{itemize}
\noindent 多进程:在操作系统中,能同时运行多个任务(程序)。
\\多进程:在同一应用程序中,有多个顺序流同时执行。
\subsubsection{线程的概念模型}
\noindent 1. 虚拟的CPU,封装在java.lang.Thread类中。
\\2. CPU所执行的代码,传递给Thread类。
\\3. CPU所处理的数据,传递给Thread类。
\subsubsection{线程体}
\noindent 1. Java的线程是通过java.lang.Thread类实现的。
\\2. 每个线程都是通过某个特定Thread对象的方法run()来完成其操作,
方法run()成为线程体。
\subsubsection{构造线程的两种方法}
\begin{enumerate}
	\item 定义一个线程类,它继承Thread并重写其中的方法run();
	\item 通过实现接口Runnable的类作为线程的目标对象,在初始化一个
	Thread类或者Thread子类的线程对象时,把目标对象传递这个线程实例,目标对象提供线程体run().
	\subitem public Thread(ThreadGroup group, Runnable target, String name);
\end{enumerate}
\subsubsection{题目}
\begin{enumerate}
	\item Java中的第一个线程都属于某个线程组;
	\item 线程只能在其创建时设置所属的线程组;
	\item 线程创建之后,不可以从一个线程组转移到另一个线程组;
	\item 新建的线程默认情况下属于其父线程所属的线程组;
	\item 线程组成:程序计数器、堆栈、栈指针,进程地址空间中的代码不属于线程。
	\item Java的线程模型:
	代码可以与其他线程共享,
	数据可以被多个线程共享,
	线程模型在java.lang.Thread类中被定义。
\end{enumerate}
\section{通过Thread类创建线程}
\subsection{Thread类}
1. Thread直接继承了Object类,并实现了Runnable接口,位于lang.lang。
\par 2. 封装了线程对象需要的属性和方法。
\subsection{创建线程}
\noindent 1. 从Thread类派生一个子类,并创建子类的对象;
\\2. 子类应该重写Thread类的run()方法,写入需要在新线程中执行的语句段;
\\3. 调用start方法来启动新线程,自动进入run方法。
\\在main中调用新的线程,注:
\begin{itemize}
	\item main方法调用thread.start()方法启动新线程后并不等待其run()方法返回就继续运行,
	线程的run方法在一边独自运行,不影响原来的main方法运行。
\end{itemize}
\subsubsection{问题}
\begin{enumerate}
	\item 一个线程是一个Thread类的实例,线程从传递给纯种的Runnable实例run()方法开始执行;
	\item 线程操作的数据来自Runnable实例,新建的线程调用start()方法不能立即进入运行状态,进入就绪状态,必须还要保证当前线程获取到资源以后调用run()才能开始执行;
	\item 在线程A中执行线程B的join()方法,则线程A等待直到B执行完成;
	\item 线程A通过调用interrupt()方法来中断其阻塞状态;
	\item 若线程A调用方法isAlive()返回值为true,则说明A正在执行中;
	\item currentThread()获取当前线程对象,而不是引用;
	\item 程序的执行完毕与超级线程(daemon threads)无关。
\end{enumerate}
\section{线程的休眠}
\subsection{延长主线程}
使用Thread.sleep(1),休眠1ms;在主线程休眠时,会执行子线程。
\par 对于线程,休眠时会把执行权交给其他线程,如果在main线程中调用多个线程,
会导致当前线程按照start顺序执行。
\par 休眠原因:让其他线程得到执行的机会。
\subsection{问题}
\begin{enumerate}
	\item Thread.sleep(cnt);会导致抛出异常;
\end{enumerate}
\section{Thread类详解}
常用方法见表1.1
\begin{table}[!h]
	\centering
	\caption{Thread类常用API方法}
	\begin{tabular}{|p{0.4\textwidth}|p{0.6\textwidth}|}%p{2cm}<{\centering} 
		\hline
		名称&说明\\
		\hline
		public Thread()&构造一个新的线程对象,默认名为Thread-n,n是从0开始递增的整数\\
		\hline
		public Thread(Runnable target)&构造一个新的线程对象,以一个实现Runnable接口
		的类的对象为参数。默认名为Thread-n,n是从0开始递增的整数。\\
		\hline
		public Thread(String name)&构造一个新的线程对象,并同时指定线程名\\
		\hline
		public static Thread currentThread()&返回当前正在运行的线程对象\\
		\hline
		public static void yield()&使当前线程对象暂停,允许别的线程开始运行\\
		\hline
		public static void sleep(long millis)&使当前线程暂停运行指定毫秒数,
		但线程并不失去已获得的锁\\
		\hline
		public void start()&启动线程,JVM将调用此线程的run()方法,结果是将
		同时运行两个线程,当前线程和执行run方法的线程\\
		\hline
		public void run()&Thread的子类应该重写此方法,内容应为该线程应执行的任务\\
		\hline
		public final void stop()&停止线程运行,释放该线程占用的对象锁\\
		\hline
		public void interrupt()&中断此线程\\
		\hline
		public fianl void join()&如果此前启动了线程A,调用join方法将等待线程A死亡
		才能继续执行当前线程\\
		\hline
		public final void join(long millis)&如果此前启动了线程A,调用join方法将等待
		指定毫秒数或线程A死亡才能继续执行当前线程\\
		\hline
		public final void setPriority(int new Priority)&设置线程优先级\\
		\hline
		public final void setDaemon(Boolean on)&设置是否为后台进程,如果当前运行线程均为后台线程,则JVM停止运行,此方法必须在start()方法
		调用前使用\\
		\hline
		public final void checkAccess()&判断当前线程是否有权力修改调用此方法的线程\\
		\hline
		public void setName(String name)&更改本线程的名称为指定参数\\
		\hline
		public final boolean isAlive()&测试线程是否处于活动状态,如果线程被启动并且
		没有死亡则返回true\\
		\hline
	\end{tabular}
\end{table}
\section{通过Runnable接口创建线程}
\subsection{Runnable使用及特点}
Runnable接口只有一个run()方法,实际上Thread也是实现了Runnable接口;
\begin{itemize}
	\item 便于多个线程共享资源
	\item Java不支持多继承,如果已经继承了某个基类,便需要实现Runnable接口来生成多线程
	\item 以实现Runnable的对象为参数建立新的线程
	\item start启动线程
\end{itemize}
两种方法比较:
\begin{itemize}
	\item 使用Runnable接口
	\subitem 可以将CPU、代码和数据分开,形成清晰的模型,还可以从其他类继承
	\item 直接继承Thread类
	\subitem
	编写简单,直接继承,重写run方法,不能再从其他类继承
\end{itemize}
\subsection{问题}
\begin{enumerate}
	\item java的线程体由Thread类的run()方法定义,线程创建时已经确定了提供线程体的对象,java中每一个线程都有自己的名字
\end{enumerate}
\section{线程内部的数据共享}
用同一个实现了Runnable接口的对象作为参数创建多个线程;这些多个线程共享同一个对象中的相同的数据。
\par 使用一个Runnable类型对象创建的多个新线程,这多个新线程就共享了这个对象的私有成员。
\begin{lstlisting}[language=java]
//MyThread.java
public class MyThread implements Runnable{

	private int sleeptime;
	
	public MyThread() {
		sleeptime = (int) (Math.random() * 5000);
	}
	
	@Override
	public void run() {
		try {
			System.out.println(Thread.currentThread().getName() + " sleep time: " + sleeptime);
			Thread.sleep(sleeptime);
		} catch (Exception ex) {
		} finally {
			System.out.println(Thread.currentThread().getName() + " finished");
		}
	}
}

//MyThreadTest.java
public class MyThreadTest {
	public static void main(String[] args) {
		MyThread mythread = new MyThread();
		System.out.println("main thread start");
		new Thread(mythread, "Thread 1").start();
		new Thread(mythread, "Thread 2").start();
		new Thread(mythread, "Thread 3").start();
		System.out.println("main thread end");
	}
}

//output
main thread start
main thread end
Thread 1 sleep time: 3750
Thread 2 sleep time: 3750
Thread 3 sleep time: 3750
Thread 2 finished
Thread 3 finished
Thread 1 finished
\end{lstlisting}