\chapter{基础}
\section{基础编程模型}
尽量减少对Java语言的依赖描述算法。
\begin{itemize}
	\item 程序是算法精确、优雅和完全的描述;
\end{itemize}
\subsection{Java程序的基本结构}
Java的基本结构:
\begin{enumerate}
	\item 原始数据类型:浮点数、整数、布尔值、字符型;
	\item 语句:声明、赋值、条件、循环、调用、返回;
	\item 数组:多个同种数据类型的集合;
	\item 静态方法:封装重用代码;
	\item 字符串:一连串的字符;
	\item 标准输入/输出;
	\item 数据抽象:数据抽象和重用代码。
\end{enumerate}
\subsection{原始数组类型与表达式}
\begin{itemize}
	\item 标识符:有字母、数字、下划线和\$组成的字符串,首字母不能是数字;
	\item 逻辑运算符:优先级! > \&\& > ||;
	\item 强制转换double转int为截断,而不是四舍五入;
	\item 别名:引用;
	\begin{lstlisting}
int[] a = new int[N];
a[i] = 1234;
int[] b = a;
b[i] = 5678;//a[i]也变化
	\end{lstlisting}
\end{itemize}
\subsection{方法的性质}
\begin{enumerate}
	\item 方法的参数按值传递:方法中改变参数的值,而不是参数本身;
	\item 方法名可以被重载;
	\item 方法只能返回一个值;
	\item 方法可以产生副作用:void类型的静态方法会产生副作用。
\end{enumerate}
\subsection{问题}
\begin{enumerate}
	\item Java的字节码:Java程序的低级表示,可以运行于JVM。
	\item Java不会自动检测溢出,但是JDK1.8中封装了安全的四则运算。
	\item Math.abs(-2147483648)=2147483648
	\item 无穷大:Double.POSITIVE\_INFINITY和Double.NEGATIVE\_INFINITY
	\item Java未初始化使用变量抛出异常;
	\item 1/0抛出除零异常,1.0/0.0为无穷大;
	\item if <expr1> if <expr2> <stmntA> else <stmntB>有二义性;
	\item Java不能重新读入标准输入中的值;
	\item 标准输入为空后,会得到一个错误,StdIn.isEmpty()能够帮助检查
	是否还有可用的输入以避免错误。
\end{enumerate}
\subsection{作业}
\begin{enumerate}
	\item 7; 200.0000002; true
	\item double, 1.118; double, 10.0; boolean, true; String, 33
	\item 算法如下:
	\begin{lstlisting}
public static void main(String[] args) {
	if (args.length != 3) {
		System.err.println("not equal");
		System.exit(1);
	}
	if (args[0].equals(args[1]) && args[0].equals(args[2])) {
		System.out.println("equal");
	} else {
		System.out.println("not equal");
	}
}
	\end{lstlisting}
	\item then非关键字;a>b没有括号;正确;c=0分号分割
	\item 算法如下:
	\begin{lstlisting}
double x, y;
x = StdIn.readDouble();
y = StdIn.readDouble();
if ( x < 1 && x > 0 && y < 1 && y > 0) {
	StdOut.print(true);
} else {
	StdOut.print(false);
}
	\end{lstlisting}
	\item 0 1 1 2 3 5 8 13 21 34 55 89 144 233 377 610 
	\item 结果如下:
	\begin{itemize}
		\item 3.00009(这里使用$a_{i+1} = \frac{a_i + \frac{n}{a_i}}{2}$)
		\item 499500
		\item 999000
	\end{itemize}
	\item b; 197; e
	\item 代码如下:
	\begin{lstlisting}
String s = "";
for (int n = N; n > 0; n /= 2)
	s = (n % 2) + s
	\end{lstlisting}
	\item 编译错误
	\begin{lstlisting}
java.lang.Error: Unresolved compilation problem: 
	The local variable a may not have been initialized
	\end{lstlisting}
	\item 代码如下:
	\begin{lstlisting}
boolean[][] dim = { {true, true}, {false, true}};
for (int i = 0; i < dim.length; i++) {
	for (int j = 0; j < dim[i].length; j++) {
		StdOut.print(dim[i][j] ? '*' : " ");
	}
	StdOut.print("\n");
}
	\end{lstlisting}
	\item 打印a[i]:0 1 2 3 4 4 3 2 1 0 ;
	\\打印i:0 1 2 3 4 5 6 7 8 9 
	\item 算法如下:
	\begin{lstlisting}
int[][] b = new int[N][M];
for (int i = 0; i < M; i++) {
	for (int j = 0; j < N; j++) {
		b[j][i] = a[i][j];
	}
}
	\end{lstlisting}
	\item 算法如下:
	\begin{lstlisting}
public static int lg(int N) throws Exception {
	boolean isPositive = true;
	if (N == 0) {
		throw new Exception("Error: N is 0");
	}
	if (N < 0) {
		isPositive = false;
		N = -N;
	}
	int result = 0;
	int sum = 2;
	while (sum <= N) {
		sum *= 2;
		result++;
	}
	return result;
}
	\end{lstlisting}
	\item 代码如下:
	\begin{lstlisting}
public static int[] histogram(int[] a, int M) {
	int[] r = new int[M];
	for (int i:a) {
		if (i <= M && i > 0) {
			r[i - 1]++;
		}
	}
	return r;
}
	\end{lstlisting}
	\item 311361142246
	\item 错误。
	\item 50,33,a*b,$2^{25}$和$3^{11}$
	\item 算法如下:
	\begin{lstlisting}
public static BigInteger[] a = new BigInteger[100];
public static void main(String[] args) {
	a[0] = BigInteger.valueOf(0);
	a[1] = BigInteger.valueOf(1);
	for (int i = 2;i < 100; i++) {
		a[i] = a[i-2].add(a[i-1]);
	}
	for (int N = 0; N < 100; N++) {
		StdOut.println(N + 1 + "\t" + a[N]);
	}
}
	\end{lstlisting}
	\item 算法如下:
	\begin{lstlisting}
public static double getLn(int n) {
if (n == 1) {
	return 0;
}
	return Math.log(n) + getLn(n-1);
}
	\end{lstlisting}
	\item 算法如下:
	\begin{lstlisting}
public static void getScore() {
	String name = StdIn.readString();
	int a = StdIn.readInt();
	int b = StdIn.readInt();
	double c = (double) a / b;
	StdOut.printf("%s\t%d\t%d\t%.3f\n", name, a, b, c);
}
	\end{lstlisting}
	\item 算法如下:
	\begin{lstlisting}
public static int rank(int key, int[] a, int lo, int hi, int depth) {
	if (lo > hi) return -1;
	int mid = lo + (hi - lo) / 2;
	if (key < a[mid]) {
		return rank(key, a, lo, mid - 1, ++depth);
	} else if (key > a[mid]) {
		return rank(key, a, mid + 1, hi, ++depth);
	} else {
		return mid;
	}
}
	\end{lstlisting}
	\item 算法实现:
	\begin{lstlisting}
while (!StdIn.isEmpty()) {
	int key = StdIn.readInt();
	int found = rank(key, whitelist);
	if ('+' == symbol && found == -1)
		StdOut.println(key);
	if ('-' == symbol && found != -1)
		StdOut.println(key);
}
	\end{lstlisting}
	\item 
\end{enumerate}

\section{数据抽象}
数据类型指的是一组值和一组对这些值的操作的集合。
\par 抽象数据类型(ADT)是一种能够对使用者隐藏数据表示的数据类型。与Java实现抽象数据类型不同,ADT实现了数据和函数之间的关联。
\subsection{使用抽象数据类型}
\subsubsection*{抽象数据类型的API}
应用程序编程接口API说明了抽象数据类型的行为,列出所有构造函数和实例方法。
\subsubsection*{对象}
对象:能够承载数据类型的值的实体。
对象三大特性:状态、标识和行为。
\subsubsection*{创建对象}
使用new实例化一个对象,new()发生的动作:
\begin{itemize}
	\item 为新的对象分配内存空间
	\item 调用构造函数初始化对象中的值
	\item 返回该对象的一个引用
\end{itemize}
\subsubsection*{数组也是对象}
在Java中,所有非原始数据类型的值都是对象,数组也是对象。当使用数组作为参数或放在赋值语句右侧时,都是创建数组引用的副本,而不是数组的副本。
\par 对于使用对象的引用,如果对象非常大,那么效率会大大提高,如果对象很小,每次获取信息都通过引用反而会降低效率。
\subsection{抽象数据类型}
\subsubsection*{作用域}
Java三种变量:参数变量、局部变量、实例变量。