\chapter{全新Web数据库}
\section{为互联网而生}
2007年,由10gen开发,设计目标:极简、灵活、作为Web应用栈的一部分。

\section{MongoDB键特性}
\subsection{文档数据模型}
MongoDB面向文档,JSON格式文档。
\par 从内部来将,MongoDB以二进制JSON格式存档文档数据——BSON,当查询MongoDB并返回结果时,这些数据转换为易于阅读的数据格式。
\par 关系型数据库包含表,MongoDB拥有集合。
\subsubsection*{无schema模型的优点}
\begin{itemize}
	\item 应用程序的代码强制数据结构而不是数据库,频繁修改数据定义的时候
	可以加速应用开发。
	\item 允许用户使用真正的变量属性来表示数据。
\end{itemize}
\par 文档建模不需要关联,可以动态地添加新的属性。
\subsection{ad hoc查询}
主动查询模式(ad hoc queries)是指不需要事先定义系统接收何种查询。