\chapter{Prototype模式}
\section{Prototype模式的概念}
\subsection{定义}
原型(Prototype)模式的定义如下:用一个已经创建的实例作为原型,通过复制该原型对象来创建一个和原型相同或相似的新对象。
\subsection{模型分析}
在原型模式结构中定义了一个抽象原型类,所有的Java类都继承自java.lang.Object,而Object类提供一个clone()方法,可以将一个Java对象复制一份。因此在Java中可以直接使用Object提供的clone()方法来实现对象的克隆,Java语言中的原型模式实现很简单。
\par
能够实现克隆的Java类必须实现一个标识接口Cloneable,表示这个Java类支持复制。如果一个类没有实现这个接口但是调用了clone()方法,Java编译器将抛出一个CloneNotSupportedException异常。
\par java.lang.Cloneable 只是起到告诉程序可以调用clone方法的作用,它本身并没有定义任何方法。
在使用原型模式克隆对象时,根据其成员对象是否也克隆,原型模式可以分为两种形式:深克隆 和 浅克隆 。
\subsection{优点}
\begin{enumerate}
	\item 当创建新的对象实例较为复杂时,使用原型模式可以简化对象的创建过程,通过一个已有实例可以提高新实例的创建效率。
	\item 可以动态增加或减少产品类。
	\item 原型模式提供了简化的创建结构。
	\item 可以使用深克隆的方式保存对象的状态。
\end{enumerate}
\subsection{缺点}
\begin{enumerate}
	\item 需要为每一个类配备一个克隆方法,而且这个克隆方法需要对类的功能进行通盘考虑,这对全新的类来说不是很难,但对已有的类进行改造时,不一定是件容易的事,必须修改其源代码,违背了“开闭原则”。
	\item 在实现深克隆时需要编写较为复杂的代码。
\end{enumerate}
\section{Java中的clone方法}
\subsection{创建对象}
\begin{enumerate}
	\item 使用new创建;
	\item 使用clone复制一个对象。
\end{enumerate}
二者区别:
\begin{enumerate}
	\item new分配内存,执行到new时,首先看new后面的操作类型,然后按照类型需要的空间分配,
	分配完内存后,调用构造函数,填充对象各个域(初始化),构造方法返回后,对象创建完毕,
	可以把它的引用(地址)发布到外部;
	\item clone分配的内存与原对象相同,然后再使用原对象中的各个域初始化clone的对象,然后
	clone方法返回,一个新的相同的对象被创建,发布到外部。
\end{enumerate}
\subsection{深拷贝和浅拷贝}
\noindent 浅拷贝:直接将原对象的属性引用值拷贝到新对象的属性字段;
\\ 深拷贝:根据原对象的属性创建一个新的相同的字符串对象,将新字符串的引用赋给
新拷贝的对象的属性字段。
\\ 注:clone默认是浅拷贝,可以覆盖实现深拷贝。
\section{角色}
\begin{itemize}
	\item Prototype(抽象原型类):Product角色负责定义用于复制现有实例来生成新实例的方法。
	\item ConcretePrototype(具体原型类):ConcretePrototype角色负责实现复制现有实例并生成新实例的方法。
	\item Client(使用者):Client角色负责使用复制实例的方法生成新的实例。
\end{itemize}
\section{原型模式——例一}
\begin{lstlisting}
//抽象原型类
public interface Product extends Cloneable{
	public abstract void use(String s);
	public abstract Product createClone();
}
\end{lstlisting}
\begin{lstlisting}
//具体原型类
public class MessageBox implements Product {
	private char decochar;
	public MessageBox(char decochar) {
		this.decochar = decochar;
	}
	public void use(String s) {
		int length = s.getBytes().length;
		for (int i = 0; i < length + 4; i++) {
			System.out.print(decochar);
		}
		System.out.println("");
		System.out.println(decochar + " " + s + " " + decochar);
		for (int i = 0; i < length + 4; i++) {
			System.out.print(decochar);
		}
		System.out.println("");
	}
	public Product createClone() {
		Product p = null;
		try {
			p = (Product) clone();
		} catch (CloneNotSupportedException e) {
			e.printStackTrace();
		}
		return p;
	}
}
public class UnderlinePen implements Product {
	private char ulchar;
	public UnderlinePen(char ulchar) {
		this.ulchar = ulchar;
	}
	@Override
	public void use(String s) {
		int length = s.getBytes().length;
		System.out.println("\"" + s + "\"");
		System.out.print(" ");
		for (int i = 0; i < length; i++) {
			System.out.print(ulchar);
		}
		System.out.println("");
	}
	
	@Override
	public Product createClone() {
		Product p = null;
		try {
			p = (Product) clone();
		} catch (CloneNotSupportedException e) {
			e.printStackTrace();
		}
		return p;
	}
}
\end{lstlisting}
\begin{lstlisting}
//客户使用类
public class Manager {
	private HashMap showcase = new HashMap();
	public void register(String name, Product proto) {
		showcase.put(name, proto);
	}
	public Product create(String protoname) {
		Product p = (Product) showcase.get(protoname);
		return p.createClone();
	}
}
\end{lstlisting}
\begin{lstlisting}
//测试类
public class Main {
	public static void main(String[] args) {
		Manager manager = new Manager();
		UnderlinePen upen = new UnderlinePen('~');
		MessageBox mbox = new MessageBox('*');
		MessageBox sbox = new MessageBox('/');
		manager.register("strong message", upen);
		manager.register("warning box", mbox);
		manager.register("slash box", sbox);
		Product p1 = manager.create("strong message");
		p1.use("Hello world.");
		Product p2 = manager.create("warning box");
		p2.use("Hello world");
		Product p3 = manager.create("slash box");
		p3.use("Hello world");
	}
}
\end{lstlisting}
\begin{lstlisting}
//output
"Hello world."
~~~~~~~~~~~~
***************
* Hello world *
***************
///////////////
/ Hello world /
///////////////
\end{lstlisting}
\section{原型模式——例二}
\begin{lstlisting}
//具体原型类
public class Realizetype implements Cloneable {
	String name;
	public Realizetype(String s) {
		name = s;
		System.out.println("具体原型创建成功!");
	}
	protected Object clone() throws CloneNotSupportedException {
		System.out.println("具体原型复制成功!");
		return (Realizetype)super.clone();
	}
}
\end{lstlisting}
\begin{lstlisting}
//测试类
public class PrototypeTest {
	public static void main(String[] args) throws CloneNotSupportedException {
		Realizetype obj1 = new Realizetype("Hello");
		Realizetype obj2 = (Realizetype) obj1.clone();
		System.out.println("obj1==obj2 " + (obj1 == obj2));
		obj2.name = "world";
		System.out.println(obj1.name);
	}
}
\end{lstlisting}
\begin{lstlisting}
//output
具体原型创建成功!
具体原型复制成功!
obj1==obj2 false
Hello
\end{lstlisting}
\section{扩展思路}
\begin{enumerate}
	\item 为什么不根据类生成实例?
	\begin{enumerate}
		\item 对象种类繁多,无法将它们整合到一个类中时;
		\item 难以根据类生成实例;
		\item 想解耦框架与生成的实例时:一旦代码中出现了要是用的类的名字,就无法
		与该类分离开来,无法实现复用。
	\end{enumerate}
\end{enumerate}
\section{其他的设计模式}
\begin{enumerate}
	\item Flyweight可以在不同的地方使用同一个实例;
	Prototype可以生成一个与当前实例的状态完全相同的实例。
	\item Memento可以保存当前实例的状态,以实现快照和撤销功能;
	Prototype可以生成一个与当前实例状态完全相同的实例;
	\item Composite和Decorator需要能够动态地创建复杂结构的实例,这里可以使用
	Prototype。
	\item Command中出现的命令可以使用Prototype。
\end{enumerate}
注:
\begin{enumerate}
	\item clone方法在java.lang.Object中;
	\item Cloneable为一个标记接口,没有方法,只是用来标记可复制的;
	\item clone浅复制只是将复制实例的字段值直接复制到新的实例中,没有考虑字段中所保存内容的实例的内容。
\end{enumerate}