\chapter{引言}
\section{什么是设计模式}
Christopher Alexander:“每一个模式描述了一个在我们周围不断重复发生的问题,
以及该问题的解决方案的核心。这样,你就能一次又一次地使用该方案而不必做重复劳动”。
\par 设计模式的四个基本要素:
\begin{enumerate}
	\item 模式名称(pattern name),用一两个词来描述模式的问题、解决方案和效果。
	\item 问题(problem)描述了应该在何时使用模式。它解释了设计问题和问题存在的前因后果,它可能描述了特定的设计问题,如怎样用对象表示算法等。也可能描述了导致不灵活设计的类或对象结构。
	\item 解决方案(solution)描述了设计的组成成分,它们之间的相互关系及各自的职责和协作方式。提供设计问题的抽象描述和怎样用一个具有一般意义的元素组合(类或对象组合)来解决这个问题。
	\item 效果(consequence)描述了模式应用的效果及使用模式应权衡的问题。模式效果包括它对系统的灵活性、扩充性或可移植性的影响。
\end{enumerate}
\section{Smalltalk MVC中的设计模式}
MVC包括三类对象。模型 Model是应用对象,视图 View是它在屏幕上的表示,控制器
Controller定义用户界面对用户输入的响应方式。不使用MVC,用户界面设计往往将这些对象
混在一起,而M V C则将它们分离以提高灵活性和复用性。
\par MVC的另一个特征是视图可以嵌套。
\par MVC允许你在不改变视图外观的情况下改变视图对用户输入的响应方式。
\par View-Controller关系是Strategy式的一个例子。一个策略是一个表述算法的对象。
当你想静态或动态地替换一个算法,或你有很多不同的算法,或算法中包含你想封装的复杂
数据结构,这时策略模式是非常有用的。
\par 还是用了其他模式:用来指定视图缺省控制器的Factory Method和用来增加视图滚动的Decorator,但
主要关系还是由Observer、Composite和Strategy给出。

\section{描述设计模式}
\noindent
\fbox{
	\parbox{\textwidth}{
\begin{enumerate}
	\item 模式和分类:模式名简洁地描述了模式的本质。
	\item 意图:设计模式是做什么的?它的基本原理和意图是什么?它解决的是什么样的特定设计问题?
	\item 别名:模式的其他名称。
	\item 动机:用以说明一个设计问题以及如何用模式中的类、对象来解决该问题的特定情景。
	\item 适用性:什么情况下可以使用该设计模式?该模式可用来改进哪些不良设计?你怎样识别这些情
	况?
	\item 结构:采用基于对象建模技术OMT的表示法对模式中的类进行图形描述
	\item 参与者:指设计模式中的类和/或对象以及它们各自的职责。
	\item 协作:模式的参与者怎样协作以实现它们的职责。
	\item 效果:模式怎样支持它的目标?使用模式的效果和所需做的权衡取舍?系统结构的哪些方面可
	以独立改变?
	\item 实现:实现模式时需要知道的一些提示、技术要点及应避免的缺陷,以及是否存在某些特定于
	实现语言的问题。
	\item 代码示例:代码片段。
	\item 已知应用:实际系统中发现的模式的例子。
	\item 相关模式:与这个模式紧密相关的模式有哪些?其间重要的不同之处是什么?
\end{enumerate}
	}
}
\section{组织编目}
\begin{table}[!h]
	\centering
	\begin{tabular}{|c|c|c|c|c|}
		\hline
		\multicolumn{2}{|c|}{}&\multicolumn{3}{|c|}{目的}\\
		\hline
		\multicolumn{2}{|c|}{}&创建型&结构型&行为型\\
		\hline
		范围&类&Factory Method&Adapter&\tabincell{l}{Interpreter\\Template Method}\\
		\hline
		&对象&\tabincell{l}{Abstract Factory\\Builder\\Prototype\\Singleton}
		&\tabincell{l}{Adapter\\Bridge\\Composite\\Decorator\\Facade\\Flyweight\\Proxy}
		&\tabincell{l}{Chain of Responsibility\\Command\\Iterator\\Mediator\\Mementor
		\\Observer\\State\\Strategy\\Visitor}\\
		\hline
	\end{tabular}
	\caption{设计模式分类}
\end{table}
第一是目的准则,即模式是用来完成什么工作的。
模式分为:创建型(Creational)、结构型(Structural)和行为型(Behavioral)。
\begin{itemize}
	\item 创建型模式与对象的创建有关;
	\item 结构型模式处理类或对象的组合;
	\item 行为型模式对类或对象怎样交互和怎样分配职责进行描述。
\end{itemize}
第二是范围准则,指定模式主要是用于类还是用于对象。
\begin{itemize}
	\item 类模式处理类和子类之间的关系,这些关系通过继承建立,是静态的,在编译时刻便确定下来了。
	\item 对象模式处理对象间的
	关系,这些关系在运行时刻是可以变化的,更具动态性。从某种意义上来说,几乎所有模式都使用继承机制,所以“类模式”只指那些集中于处理类间关系的模式,而大部分模式都属于对象模式的范畴。
\end{itemize}
\section{设计模式怎样解决设计问题}
\subsection{寻找合适的对象}
面向对象程序由对象组成,对象包括数据和对数据进行操作的过程,过程通常称为方法
或操作。
\par 面向对象设计最困难的部分是将系统分解成对象集合。因为要考虑许多因素:封装、粒
度、依赖关系、灵活性、性能、演化、复用等等,它们都影响着系统的分解,并且这些因素
通常还是互相冲突的。
\subsection{决定对象的粒度}
对象在大小和数目上变化极大。它们能表示下自硬件或上自整个应用的任何事物。
\subsection{指定对象接口}
对象声明的每一个操作指定操作名、作为参数的对象和返回值,这就是所谓的操作的型
构。对象操作所定义的所有操作型构的集合被称为该对象的接口。对象
接口描述了该对象所能接受的全部请求的集合,任何匹配对象接口中型构的请求都可以发送
给该对象。
\\ 动态绑定:发送给对象的请求和它的相应操作在运行时刻的连接就称之为动态绑定。
\subsection{描述对象的实现}
对象的实现是由它的类决定的,类指定了对象的内部数据和表示,也定义了对象所能完成的操作.
\\ 抽象类主要目的是为它的子类定义公共接口。一个抽象类将把它的部分
或全部操作的实现延迟到子类中,因此,一个抽象类不能被实例化。
\\ 混入类是给其他类提供可选择的接口或功能的类。它与抽象类一样不能实例
化。混入类要求多继承。
\\ 只根据抽象类中定义的接口来操纵对象有以下两个好处:
\begin{itemize}
	\item 客户无须知道他们使用对象的特定类型,只须对象有客户所期望的接口。
	\item 客户无须知道他们使用的对象是用什么类来实现的,他们只须知道定义接口的抽象类。
\end{itemize}